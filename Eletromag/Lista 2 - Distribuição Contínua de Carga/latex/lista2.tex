\documentclass[11pt]{article}

    \usepackage[breakable]{tcolorbox}
    \usepackage{parskip} % Stop auto-indenting (to mimic markdown behaviour)
    
    \usepackage{iftex}
    \ifPDFTeX
    	\usepackage[T1]{fontenc}
    	\usepackage{mathpazo}
    \else
    	\usepackage{fontspec}
    \fi

    % Basic figure setup, for now with no caption control since it's done
    % automatically by Pandoc (which extracts ![](path) syntax from Markdown).
    \usepackage{graphicx}
    % Maintain compatibility with old templates. Remove in nbconvert 6.0
    \let\Oldincludegraphics\includegraphics
    % Ensure that by default, figures have no caption (until we provide a
    % proper Figure object with a Caption API and a way to capture that
    % in the conversion process - todo).
    \usepackage{caption}
    \DeclareCaptionFormat{nocaption}{}
    \captionsetup{format=nocaption,aboveskip=0pt,belowskip=0pt}

    \usepackage{float}
    \floatplacement{figure}{H} % forces figures to be placed at the correct location
    \usepackage{xcolor} % Allow colors to be defined
    \usepackage{enumerate} % Needed for markdown enumerations to work
    \usepackage{geometry} % Used to adjust the document margins
    \usepackage{amsmath} % Equations
    \usepackage{amssymb} % Equations
    \usepackage{textcomp} % defines textquotesingle
    % Hack from http://tex.stackexchange.com/a/47451/13684:
    \AtBeginDocument{%
        \def\PYZsq{\textquotesingle}% Upright quotes in Pygmentized code
    }
    \usepackage{upquote} % Upright quotes for verbatim code
    \usepackage{eurosym} % defines \euro
    \usepackage[mathletters]{ucs} % Extended unicode (utf-8) support
    \usepackage{fancyvrb} % verbatim replacement that allows latex
    \usepackage{grffile} % extends the file name processing of package graphics 
                         % to support a larger range
    \makeatletter % fix for old versions of grffile with XeLaTeX
    \@ifpackagelater{grffile}{2019/11/01}
    {
      % Do nothing on new versions
    }
    {
      \def\Gread@@xetex#1{%
        \IfFileExists{"\Gin@base".bb}%
        {\Gread@eps{\Gin@base.bb}}%
        {\Gread@@xetex@aux#1}%
      }
    }
    \makeatother
    \usepackage[Export]{adjustbox} % Used to constrain images to a maximum size
    \adjustboxset{max size={0.9\linewidth}{0.9\paperheight}}

    % The hyperref package gives us a pdf with properly built
    % internal navigation ('pdf bookmarks' for the table of contents,
    % internal cross-reference links, web links for URLs, etc.)
    \usepackage{hyperref}
    % The default LaTeX title has an obnoxious amount of whitespace. By default,
    % titling removes some of it. It also provides customization options.
    \usepackage{titling}
    \usepackage{longtable} % longtable support required by pandoc >1.10
    \usepackage{booktabs}  % table support for pandoc > 1.12.2
    \usepackage[inline]{enumitem} % IRkernel/repr support (it uses the enumerate* environment)
    \usepackage[normalem]{ulem} % ulem is needed to support strikethroughs (\sout)
                                % normalem makes italics be italics, not underlines
    \usepackage{mathrsfs}
    

    
    % Colors for the hyperref package
    \definecolor{urlcolor}{rgb}{0,.145,.698}
    \definecolor{linkcolor}{rgb}{.71,0.21,0.01}
    \definecolor{citecolor}{rgb}{.12,.54,.11}

    % ANSI colors
    \definecolor{ansi-black}{HTML}{3E424D}
    \definecolor{ansi-black-intense}{HTML}{282C36}
    \definecolor{ansi-red}{HTML}{E75C58}
    \definecolor{ansi-red-intense}{HTML}{B22B31}
    \definecolor{ansi-green}{HTML}{00A250}
    \definecolor{ansi-green-intense}{HTML}{007427}
    \definecolor{ansi-yellow}{HTML}{DDB62B}
    \definecolor{ansi-yellow-intense}{HTML}{B27D12}
    \definecolor{ansi-blue}{HTML}{208FFB}
    \definecolor{ansi-blue-intense}{HTML}{0065CA}
    \definecolor{ansi-magenta}{HTML}{D160C4}
    \definecolor{ansi-magenta-intense}{HTML}{A03196}
    \definecolor{ansi-cyan}{HTML}{60C6C8}
    \definecolor{ansi-cyan-intense}{HTML}{258F8F}
    \definecolor{ansi-white}{HTML}{C5C1B4}
    \definecolor{ansi-white-intense}{HTML}{A1A6B2}
    \definecolor{ansi-default-inverse-fg}{HTML}{FFFFFF}
    \definecolor{ansi-default-inverse-bg}{HTML}{000000}

    % common color for the border for error outputs.
    \definecolor{outerrorbackground}{HTML}{FFDFDF}

    % commands and environments needed by pandoc snippets
    % extracted from the output of `pandoc -s`
    \providecommand{\tightlist}{%
      \setlength{\itemsep}{0pt}\setlength{\parskip}{0pt}}
    \DefineVerbatimEnvironment{Highlighting}{Verbatim}{commandchars=\\\{\}}
    % Add ',fontsize=\small' for more characters per line
    \newenvironment{Shaded}{}{}
    \newcommand{\KeywordTok}[1]{\textcolor[rgb]{0.00,0.44,0.13}{\textbf{{#1}}}}
    \newcommand{\DataTypeTok}[1]{\textcolor[rgb]{0.56,0.13,0.00}{{#1}}}
    \newcommand{\DecValTok}[1]{\textcolor[rgb]{0.25,0.63,0.44}{{#1}}}
    \newcommand{\BaseNTok}[1]{\textcolor[rgb]{0.25,0.63,0.44}{{#1}}}
    \newcommand{\FloatTok}[1]{\textcolor[rgb]{0.25,0.63,0.44}{{#1}}}
    \newcommand{\CharTok}[1]{\textcolor[rgb]{0.25,0.44,0.63}{{#1}}}
    \newcommand{\StringTok}[1]{\textcolor[rgb]{0.25,0.44,0.63}{{#1}}}
    \newcommand{\CommentTok}[1]{\textcolor[rgb]{0.38,0.63,0.69}{\textit{{#1}}}}
    \newcommand{\OtherTok}[1]{\textcolor[rgb]{0.00,0.44,0.13}{{#1}}}
    \newcommand{\AlertTok}[1]{\textcolor[rgb]{1.00,0.00,0.00}{\textbf{{#1}}}}
    \newcommand{\FunctionTok}[1]{\textcolor[rgb]{0.02,0.16,0.49}{{#1}}}
    \newcommand{\RegionMarkerTok}[1]{{#1}}
    \newcommand{\ErrorTok}[1]{\textcolor[rgb]{1.00,0.00,0.00}{\textbf{{#1}}}}
    \newcommand{\NormalTok}[1]{{#1}}
    
    % Additional commands for more recent versions of Pandoc
    \newcommand{\ConstantTok}[1]{\textcolor[rgb]{0.53,0.00,0.00}{{#1}}}
    \newcommand{\SpecialCharTok}[1]{\textcolor[rgb]{0.25,0.44,0.63}{{#1}}}
    \newcommand{\VerbatimStringTok}[1]{\textcolor[rgb]{0.25,0.44,0.63}{{#1}}}
    \newcommand{\SpecialStringTok}[1]{\textcolor[rgb]{0.73,0.40,0.53}{{#1}}}
    \newcommand{\ImportTok}[1]{{#1}}
    \newcommand{\DocumentationTok}[1]{\textcolor[rgb]{0.73,0.13,0.13}{\textit{{#1}}}}
    \newcommand{\AnnotationTok}[1]{\textcolor[rgb]{0.38,0.63,0.69}{\textbf{\textit{{#1}}}}}
    \newcommand{\CommentVarTok}[1]{\textcolor[rgb]{0.38,0.63,0.69}{\textbf{\textit{{#1}}}}}
    \newcommand{\VariableTok}[1]{\textcolor[rgb]{0.10,0.09,0.49}{{#1}}}
    \newcommand{\ControlFlowTok}[1]{\textcolor[rgb]{0.00,0.44,0.13}{\textbf{{#1}}}}
    \newcommand{\OperatorTok}[1]{\textcolor[rgb]{0.40,0.40,0.40}{{#1}}}
    \newcommand{\BuiltInTok}[1]{{#1}}
    \newcommand{\ExtensionTok}[1]{{#1}}
    \newcommand{\PreprocessorTok}[1]{\textcolor[rgb]{0.74,0.48,0.00}{{#1}}}
    \newcommand{\AttributeTok}[1]{\textcolor[rgb]{0.49,0.56,0.16}{{#1}}}
    \newcommand{\InformationTok}[1]{\textcolor[rgb]{0.38,0.63,0.69}{\textbf{\textit{{#1}}}}}
    \newcommand{\WarningTok}[1]{\textcolor[rgb]{0.38,0.63,0.69}{\textbf{\textit{{#1}}}}}
    
    
    % Define a nice break command that doesn't care if a line doesn't already
    % exist.
    \def\br{\hspace*{\fill} \\* }
    % Math Jax compatibility definitions
    \def\gt{>}
    \def\lt{<}
    \let\Oldtex\TeX
    \let\Oldlatex\LaTeX
    \renewcommand{\TeX}{\textrm{\Oldtex}}
    \renewcommand{\LaTeX}{\textrm{\Oldlatex}}
    % Document parameters
    % Document title
    \title{lista2}
    
    
    
    
    
% Pygments definitions
\makeatletter
\def\PY@reset{\let\PY@it=\relax \let\PY@bf=\relax%
    \let\PY@ul=\relax \let\PY@tc=\relax%
    \let\PY@bc=\relax \let\PY@ff=\relax}
\def\PY@tok#1{\csname PY@tok@#1\endcsname}
\def\PY@toks#1+{\ifx\relax#1\empty\else%
    \PY@tok{#1}\expandafter\PY@toks\fi}
\def\PY@do#1{\PY@bc{\PY@tc{\PY@ul{%
    \PY@it{\PY@bf{\PY@ff{#1}}}}}}}
\def\PY#1#2{\PY@reset\PY@toks#1+\relax+\PY@do{#2}}

\expandafter\def\csname PY@tok@w\endcsname{\def\PY@tc##1{\textcolor[rgb]{0.73,0.73,0.73}{##1}}}
\expandafter\def\csname PY@tok@c\endcsname{\let\PY@it=\textit\def\PY@tc##1{\textcolor[rgb]{0.25,0.50,0.50}{##1}}}
\expandafter\def\csname PY@tok@cp\endcsname{\def\PY@tc##1{\textcolor[rgb]{0.74,0.48,0.00}{##1}}}
\expandafter\def\csname PY@tok@k\endcsname{\let\PY@bf=\textbf\def\PY@tc##1{\textcolor[rgb]{0.00,0.50,0.00}{##1}}}
\expandafter\def\csname PY@tok@kp\endcsname{\def\PY@tc##1{\textcolor[rgb]{0.00,0.50,0.00}{##1}}}
\expandafter\def\csname PY@tok@kt\endcsname{\def\PY@tc##1{\textcolor[rgb]{0.69,0.00,0.25}{##1}}}
\expandafter\def\csname PY@tok@o\endcsname{\def\PY@tc##1{\textcolor[rgb]{0.40,0.40,0.40}{##1}}}
\expandafter\def\csname PY@tok@ow\endcsname{\let\PY@bf=\textbf\def\PY@tc##1{\textcolor[rgb]{0.67,0.13,1.00}{##1}}}
\expandafter\def\csname PY@tok@nb\endcsname{\def\PY@tc##1{\textcolor[rgb]{0.00,0.50,0.00}{##1}}}
\expandafter\def\csname PY@tok@nf\endcsname{\def\PY@tc##1{\textcolor[rgb]{0.00,0.00,1.00}{##1}}}
\expandafter\def\csname PY@tok@nc\endcsname{\let\PY@bf=\textbf\def\PY@tc##1{\textcolor[rgb]{0.00,0.00,1.00}{##1}}}
\expandafter\def\csname PY@tok@nn\endcsname{\let\PY@bf=\textbf\def\PY@tc##1{\textcolor[rgb]{0.00,0.00,1.00}{##1}}}
\expandafter\def\csname PY@tok@ne\endcsname{\let\PY@bf=\textbf\def\PY@tc##1{\textcolor[rgb]{0.82,0.25,0.23}{##1}}}
\expandafter\def\csname PY@tok@nv\endcsname{\def\PY@tc##1{\textcolor[rgb]{0.10,0.09,0.49}{##1}}}
\expandafter\def\csname PY@tok@no\endcsname{\def\PY@tc##1{\textcolor[rgb]{0.53,0.00,0.00}{##1}}}
\expandafter\def\csname PY@tok@nl\endcsname{\def\PY@tc##1{\textcolor[rgb]{0.63,0.63,0.00}{##1}}}
\expandafter\def\csname PY@tok@ni\endcsname{\let\PY@bf=\textbf\def\PY@tc##1{\textcolor[rgb]{0.60,0.60,0.60}{##1}}}
\expandafter\def\csname PY@tok@na\endcsname{\def\PY@tc##1{\textcolor[rgb]{0.49,0.56,0.16}{##1}}}
\expandafter\def\csname PY@tok@nt\endcsname{\let\PY@bf=\textbf\def\PY@tc##1{\textcolor[rgb]{0.00,0.50,0.00}{##1}}}
\expandafter\def\csname PY@tok@nd\endcsname{\def\PY@tc##1{\textcolor[rgb]{0.67,0.13,1.00}{##1}}}
\expandafter\def\csname PY@tok@s\endcsname{\def\PY@tc##1{\textcolor[rgb]{0.73,0.13,0.13}{##1}}}
\expandafter\def\csname PY@tok@sd\endcsname{\let\PY@it=\textit\def\PY@tc##1{\textcolor[rgb]{0.73,0.13,0.13}{##1}}}
\expandafter\def\csname PY@tok@si\endcsname{\let\PY@bf=\textbf\def\PY@tc##1{\textcolor[rgb]{0.73,0.40,0.53}{##1}}}
\expandafter\def\csname PY@tok@se\endcsname{\let\PY@bf=\textbf\def\PY@tc##1{\textcolor[rgb]{0.73,0.40,0.13}{##1}}}
\expandafter\def\csname PY@tok@sr\endcsname{\def\PY@tc##1{\textcolor[rgb]{0.73,0.40,0.53}{##1}}}
\expandafter\def\csname PY@tok@ss\endcsname{\def\PY@tc##1{\textcolor[rgb]{0.10,0.09,0.49}{##1}}}
\expandafter\def\csname PY@tok@sx\endcsname{\def\PY@tc##1{\textcolor[rgb]{0.00,0.50,0.00}{##1}}}
\expandafter\def\csname PY@tok@m\endcsname{\def\PY@tc##1{\textcolor[rgb]{0.40,0.40,0.40}{##1}}}
\expandafter\def\csname PY@tok@gh\endcsname{\let\PY@bf=\textbf\def\PY@tc##1{\textcolor[rgb]{0.00,0.00,0.50}{##1}}}
\expandafter\def\csname PY@tok@gu\endcsname{\let\PY@bf=\textbf\def\PY@tc##1{\textcolor[rgb]{0.50,0.00,0.50}{##1}}}
\expandafter\def\csname PY@tok@gd\endcsname{\def\PY@tc##1{\textcolor[rgb]{0.63,0.00,0.00}{##1}}}
\expandafter\def\csname PY@tok@gi\endcsname{\def\PY@tc##1{\textcolor[rgb]{0.00,0.63,0.00}{##1}}}
\expandafter\def\csname PY@tok@gr\endcsname{\def\PY@tc##1{\textcolor[rgb]{1.00,0.00,0.00}{##1}}}
\expandafter\def\csname PY@tok@ge\endcsname{\let\PY@it=\textit}
\expandafter\def\csname PY@tok@gs\endcsname{\let\PY@bf=\textbf}
\expandafter\def\csname PY@tok@gp\endcsname{\let\PY@bf=\textbf\def\PY@tc##1{\textcolor[rgb]{0.00,0.00,0.50}{##1}}}
\expandafter\def\csname PY@tok@go\endcsname{\def\PY@tc##1{\textcolor[rgb]{0.53,0.53,0.53}{##1}}}
\expandafter\def\csname PY@tok@gt\endcsname{\def\PY@tc##1{\textcolor[rgb]{0.00,0.27,0.87}{##1}}}
\expandafter\def\csname PY@tok@err\endcsname{\def\PY@bc##1{\setlength{\fboxsep}{0pt}\fcolorbox[rgb]{1.00,0.00,0.00}{1,1,1}{\strut ##1}}}
\expandafter\def\csname PY@tok@kc\endcsname{\let\PY@bf=\textbf\def\PY@tc##1{\textcolor[rgb]{0.00,0.50,0.00}{##1}}}
\expandafter\def\csname PY@tok@kd\endcsname{\let\PY@bf=\textbf\def\PY@tc##1{\textcolor[rgb]{0.00,0.50,0.00}{##1}}}
\expandafter\def\csname PY@tok@kn\endcsname{\let\PY@bf=\textbf\def\PY@tc##1{\textcolor[rgb]{0.00,0.50,0.00}{##1}}}
\expandafter\def\csname PY@tok@kr\endcsname{\let\PY@bf=\textbf\def\PY@tc##1{\textcolor[rgb]{0.00,0.50,0.00}{##1}}}
\expandafter\def\csname PY@tok@bp\endcsname{\def\PY@tc##1{\textcolor[rgb]{0.00,0.50,0.00}{##1}}}
\expandafter\def\csname PY@tok@fm\endcsname{\def\PY@tc##1{\textcolor[rgb]{0.00,0.00,1.00}{##1}}}
\expandafter\def\csname PY@tok@vc\endcsname{\def\PY@tc##1{\textcolor[rgb]{0.10,0.09,0.49}{##1}}}
\expandafter\def\csname PY@tok@vg\endcsname{\def\PY@tc##1{\textcolor[rgb]{0.10,0.09,0.49}{##1}}}
\expandafter\def\csname PY@tok@vi\endcsname{\def\PY@tc##1{\textcolor[rgb]{0.10,0.09,0.49}{##1}}}
\expandafter\def\csname PY@tok@vm\endcsname{\def\PY@tc##1{\textcolor[rgb]{0.10,0.09,0.49}{##1}}}
\expandafter\def\csname PY@tok@sa\endcsname{\def\PY@tc##1{\textcolor[rgb]{0.73,0.13,0.13}{##1}}}
\expandafter\def\csname PY@tok@sb\endcsname{\def\PY@tc##1{\textcolor[rgb]{0.73,0.13,0.13}{##1}}}
\expandafter\def\csname PY@tok@sc\endcsname{\def\PY@tc##1{\textcolor[rgb]{0.73,0.13,0.13}{##1}}}
\expandafter\def\csname PY@tok@dl\endcsname{\def\PY@tc##1{\textcolor[rgb]{0.73,0.13,0.13}{##1}}}
\expandafter\def\csname PY@tok@s2\endcsname{\def\PY@tc##1{\textcolor[rgb]{0.73,0.13,0.13}{##1}}}
\expandafter\def\csname PY@tok@sh\endcsname{\def\PY@tc##1{\textcolor[rgb]{0.73,0.13,0.13}{##1}}}
\expandafter\def\csname PY@tok@s1\endcsname{\def\PY@tc##1{\textcolor[rgb]{0.73,0.13,0.13}{##1}}}
\expandafter\def\csname PY@tok@mb\endcsname{\def\PY@tc##1{\textcolor[rgb]{0.40,0.40,0.40}{##1}}}
\expandafter\def\csname PY@tok@mf\endcsname{\def\PY@tc##1{\textcolor[rgb]{0.40,0.40,0.40}{##1}}}
\expandafter\def\csname PY@tok@mh\endcsname{\def\PY@tc##1{\textcolor[rgb]{0.40,0.40,0.40}{##1}}}
\expandafter\def\csname PY@tok@mi\endcsname{\def\PY@tc##1{\textcolor[rgb]{0.40,0.40,0.40}{##1}}}
\expandafter\def\csname PY@tok@il\endcsname{\def\PY@tc##1{\textcolor[rgb]{0.40,0.40,0.40}{##1}}}
\expandafter\def\csname PY@tok@mo\endcsname{\def\PY@tc##1{\textcolor[rgb]{0.40,0.40,0.40}{##1}}}
\expandafter\def\csname PY@tok@ch\endcsname{\let\PY@it=\textit\def\PY@tc##1{\textcolor[rgb]{0.25,0.50,0.50}{##1}}}
\expandafter\def\csname PY@tok@cm\endcsname{\let\PY@it=\textit\def\PY@tc##1{\textcolor[rgb]{0.25,0.50,0.50}{##1}}}
\expandafter\def\csname PY@tok@cpf\endcsname{\let\PY@it=\textit\def\PY@tc##1{\textcolor[rgb]{0.25,0.50,0.50}{##1}}}
\expandafter\def\csname PY@tok@c1\endcsname{\let\PY@it=\textit\def\PY@tc##1{\textcolor[rgb]{0.25,0.50,0.50}{##1}}}
\expandafter\def\csname PY@tok@cs\endcsname{\let\PY@it=\textit\def\PY@tc##1{\textcolor[rgb]{0.25,0.50,0.50}{##1}}}

\def\PYZbs{\char`\\}
\def\PYZus{\char`\_}
\def\PYZob{\char`\{}
\def\PYZcb{\char`\}}
\def\PYZca{\char`\^}
\def\PYZam{\char`\&}
\def\PYZlt{\char`\<}
\def\PYZgt{\char`\>}
\def\PYZsh{\char`\#}
\def\PYZpc{\char`\%}
\def\PYZdl{\char`\$}
\def\PYZhy{\char`\-}
\def\PYZsq{\char`\'}
\def\PYZdq{\char`\"}
\def\PYZti{\char`\~}
% for compatibility with earlier versions
\def\PYZat{@}
\def\PYZlb{[}
\def\PYZrb{]}
\makeatother


    % For linebreaks inside Verbatim environment from package fancyvrb. 
    \makeatletter
        \newbox\Wrappedcontinuationbox 
        \newbox\Wrappedvisiblespacebox 
        \newcommand*\Wrappedvisiblespace {\textcolor{red}{\textvisiblespace}} 
        \newcommand*\Wrappedcontinuationsymbol {\textcolor{red}{\llap{\tiny$\m@th\hookrightarrow$}}} 
        \newcommand*\Wrappedcontinuationindent {3ex } 
        \newcommand*\Wrappedafterbreak {\kern\Wrappedcontinuationindent\copy\Wrappedcontinuationbox} 
        % Take advantage of the already applied Pygments mark-up to insert 
        % potential linebreaks for TeX processing. 
        %        {, <, #, %, $, ' and ": go to next line. 
        %        _, }, ^, &, >, - and ~: stay at end of broken line. 
        % Use of \textquotesingle for straight quote. 
        \newcommand*\Wrappedbreaksatspecials {% 
            \def\PYGZus{\discretionary{\char`\_}{\Wrappedafterbreak}{\char`\_}}% 
            \def\PYGZob{\discretionary{}{\Wrappedafterbreak\char`\{}{\char`\{}}% 
            \def\PYGZcb{\discretionary{\char`\}}{\Wrappedafterbreak}{\char`\}}}% 
            \def\PYGZca{\discretionary{\char`\^}{\Wrappedafterbreak}{\char`\^}}% 
            \def\PYGZam{\discretionary{\char`\&}{\Wrappedafterbreak}{\char`\&}}% 
            \def\PYGZlt{\discretionary{}{\Wrappedafterbreak\char`\<}{\char`\<}}% 
            \def\PYGZgt{\discretionary{\char`\>}{\Wrappedafterbreak}{\char`\>}}% 
            \def\PYGZsh{\discretionary{}{\Wrappedafterbreak\char`\#}{\char`\#}}% 
            \def\PYGZpc{\discretionary{}{\Wrappedafterbreak\char`\%}{\char`\%}}% 
            \def\PYGZdl{\discretionary{}{\Wrappedafterbreak\char`\$}{\char`\$}}% 
            \def\PYGZhy{\discretionary{\char`\-}{\Wrappedafterbreak}{\char`\-}}% 
            \def\PYGZsq{\discretionary{}{\Wrappedafterbreak\textquotesingle}{\textquotesingle}}% 
            \def\PYGZdq{\discretionary{}{\Wrappedafterbreak\char`\"}{\char`\"}}% 
            \def\PYGZti{\discretionary{\char`\~}{\Wrappedafterbreak}{\char`\~}}% 
        } 
        % Some characters . , ; ? ! / are not pygmentized. 
        % This macro makes them "active" and they will insert potential linebreaks 
        \newcommand*\Wrappedbreaksatpunct {% 
            \lccode`\~`\.\lowercase{\def~}{\discretionary{\hbox{\char`\.}}{\Wrappedafterbreak}{\hbox{\char`\.}}}% 
            \lccode`\~`\,\lowercase{\def~}{\discretionary{\hbox{\char`\,}}{\Wrappedafterbreak}{\hbox{\char`\,}}}% 
            \lccode`\~`\;\lowercase{\def~}{\discretionary{\hbox{\char`\;}}{\Wrappedafterbreak}{\hbox{\char`\;}}}% 
            \lccode`\~`\:\lowercase{\def~}{\discretionary{\hbox{\char`\:}}{\Wrappedafterbreak}{\hbox{\char`\:}}}% 
            \lccode`\~`\?\lowercase{\def~}{\discretionary{\hbox{\char`\?}}{\Wrappedafterbreak}{\hbox{\char`\?}}}% 
            \lccode`\~`\!\lowercase{\def~}{\discretionary{\hbox{\char`\!}}{\Wrappedafterbreak}{\hbox{\char`\!}}}% 
            \lccode`\~`\/\lowercase{\def~}{\discretionary{\hbox{\char`\/}}{\Wrappedafterbreak}{\hbox{\char`\/}}}% 
            \catcode`\.\active
            \catcode`\,\active 
            \catcode`\;\active
            \catcode`\:\active
            \catcode`\?\active
            \catcode`\!\active
            \catcode`\/\active 
            \lccode`\~`\~ 	
        }
    \makeatother

    \let\OriginalVerbatim=\Verbatim
    \makeatletter
    \renewcommand{\Verbatim}[1][1]{%
        %\parskip\z@skip
        \sbox\Wrappedcontinuationbox {\Wrappedcontinuationsymbol}%
        \sbox\Wrappedvisiblespacebox {\FV@SetupFont\Wrappedvisiblespace}%
        \def\FancyVerbFormatLine ##1{\hsize\linewidth
            \vtop{\raggedright\hyphenpenalty\z@\exhyphenpenalty\z@
                \doublehyphendemerits\z@\finalhyphendemerits\z@
                \strut ##1\strut}%
        }%
        % If the linebreak is at a space, the latter will be displayed as visible
        % space at end of first line, and a continuation symbol starts next line.
        % Stretch/shrink are however usually zero for typewriter font.
        \def\FV@Space {%
            \nobreak\hskip\z@ plus\fontdimen3\font minus\fontdimen4\font
            \discretionary{\copy\Wrappedvisiblespacebox}{\Wrappedafterbreak}
            {\kern\fontdimen2\font}%
        }%
        
        % Allow breaks at special characters using \PYG... macros.
        \Wrappedbreaksatspecials
        % Breaks at punctuation characters . , ; ? ! and / need catcode=\active 	
        \OriginalVerbatim[#1,codes*=\Wrappedbreaksatpunct]%
    }
    \makeatother

    % Exact colors from NB
    \definecolor{incolor}{HTML}{303F9F}
    \definecolor{outcolor}{HTML}{D84315}
    \definecolor{cellborder}{HTML}{CFCFCF}
    \definecolor{cellbackground}{HTML}{F7F7F7}
    
    % prompt
    \makeatletter
    \newcommand{\boxspacing}{\kern\kvtcb@left@rule\kern\kvtcb@boxsep}
    \makeatother
    \newcommand{\prompt}[4]{
        {\ttfamily\llap{{\color{#2}[#3]:\hspace{3pt}#4}}\vspace{-\baselineskip}}
    }
    

    
    % Prevent overflowing lines due to hard-to-break entities
    \sloppy 
    % Setup hyperref package
    \hypersetup{
      breaklinks=true,  % so long urls are correctly broken across lines
      colorlinks=true,
      urlcolor=urlcolor,
      linkcolor=linkcolor,
      citecolor=citecolor,
      }
    % Slightly bigger margins than the latex defaults
    
    \geometry{verbose,tmargin=1in,bmargin=1in,lmargin=1in,rmargin=1in}
    
    

\begin{document}
    
    \maketitle
    
    

    
    \hypertarget{lista-2}{%
\section{Lista 2}\label{lista-2}}

\hypertarget{q1}{%
\subsection{Q1}\label{q1}}

    Primeiro passo, calculo a densidade da distribuição de carga e
configurou-se o parâmetro

    \begin{tcolorbox}[breakable, size=fbox, boxrule=1pt, pad at break*=1mm,colback=cellbackground, colframe=cellborder]
\prompt{In}{incolor}{22}{\boxspacing}
\begin{Verbatim}[commandchars=\\\{\}]
\PY{k+kn}{from} \PY{n+nn}{math} \PY{k+kn}{import} \PY{o}{*}
\PY{k+kn}{import} \PY{n+nn}{numpy} \PY{k}{as} \PY{n+nn}{np} 

\PY{n}{q1}\PY{o}{=}\PY{l+m+mf}{50e\PYZhy{}6}
\PY{n}{Q}\PY{o}{=}\PY{l+m+mf}{500e\PYZhy{}6}\PY{o}{*}\PY{n}{pi} 
\PY{n}{radius}\PY{o}{=}\PY{l+m+mi}{5}

\PY{n}{rho}\PY{o}{=}\PY{n}{Q}\PY{o}{/}\PY{p}{(}\PY{n}{pi}\PY{o}{*}\PY{n}{radius}\PY{o}{*}\PY{o}{*}\PY{l+m+mi}{2}\PY{p}{)}

\PY{n+nb}{print}\PY{p}{(}\PY{l+s+s2}{\PYZdq{}}\PY{l+s+s2}{Rho = }\PY{l+s+si}{\PYZob{}:.2e\PYZcb{}}\PY{l+s+s2}{\PYZdq{}}\PY{o}{.}\PY{n}{format}\PY{p}{(}\PY{n}{rho}\PY{p}{)}\PY{p}{)}
\end{Verbatim}
\end{tcolorbox}

    \begin{Verbatim}[commandchars=\\\{\}]
Rho = 2.00e-05
    \end{Verbatim}

    \[ \rho_s = \frac{Q}{S}\] \[ \rho_s = 2.00 * 10^{-5} \]

    Segundo passo,

\[ d\vec{F} = \frac{k*q_1*dQ}{R^3} \vec{R}\]

    onde o vetor \(\vec{R}\) é composto por:
\[ \vec{R} = -r\vec{a}_r + 5\vec{a}_z \]

    e \[ dQ = \rho_s \ r \ dr \ d\phi \]

    então ficou-se:

\[ d\vec{F} = \frac{k \ q_1 \ (\rho_s \ r \ dr \ d\phi)}{R^3} \vec{R}\]

na forma integral:

\[ \vec{F} = \iint \frac{k \ q_1 \ (\rho_s \ r \ dr \ d\phi)}{R^3} \vec{R} \]

    \begin{tcolorbox}[breakable, size=fbox, boxrule=1pt, pad at break*=1mm,colback=cellbackground, colframe=cellborder]
\prompt{In}{incolor}{23}{\boxspacing}
\begin{Verbatim}[commandchars=\\\{\}]
\PY{c+c1}{\PYZsh{} from sympy import *}
\PY{k+kn}{from} \PY{n+nn}{math} \PY{k+kn}{import} \PY{o}{*}
\PY{c+c1}{\PYZsh{} from numpy import *}
\PY{k+kn}{import} \PY{n+nn}{sympy} \PY{k}{as} \PY{n+nn}{sp}
\PY{k+kn}{import} \PY{n+nn}{numpy} \PY{k}{as} \PY{n+nn}{np}
\PY{n}{k} \PY{o}{=} \PY{l+m+mf}{9e9}

\PY{n}{r} \PY{o}{=} \PY{n}{sp}\PY{o}{.}\PY{n}{Symbol}\PY{p}{(}\PY{l+s+s1}{\PYZsq{}}\PY{l+s+s1}{r}\PY{l+s+s1}{\PYZsq{}}\PY{p}{)}
\PY{n}{p} \PY{o}{=} \PY{n}{sp}\PY{o}{.}\PY{n}{Symbol}\PY{p}{(}\PY{l+s+s1}{\PYZsq{}}\PY{l+s+s1}{p}\PY{l+s+s1}{\PYZsq{}}\PY{p}{)}
\PY{n}{R} \PY{o}{=} \PY{n}{sp}\PY{o}{.}\PY{n}{sqrt}\PY{p}{(}\PY{p}{(}\PY{l+m+mi}{25}\PY{o}{+}\PY{n}{r}\PY{o}{*}\PY{o}{*}\PY{l+m+mi}{2}\PY{p}{)}\PY{p}{)}

\PY{n}{f} \PY{o}{=} \PY{p}{(}\PY{n}{k}\PY{o}{*}\PY{n}{q1}\PY{o}{*}\PY{n}{rho}\PY{o}{*}\PY{n}{radius}\PY{o}{*}\PY{n}{r}\PY{p}{)}\PY{o}{/}\PY{p}{(}\PY{n}{R}\PY{o}{*}\PY{o}{*}\PY{l+m+mi}{3}\PY{p}{)}

\PY{n}{F} \PY{o}{=} \PY{n}{integrate}\PY{p}{(}\PY{n}{integrate}\PY{p}{(}\PY{n}{f}\PY{p}{,}\PY{p}{(}\PY{n}{r}\PY{p}{,}\PY{l+m+mi}{0}\PY{p}{,}\PY{l+m+mi}{5}\PY{p}{)}\PY{p}{)}\PY{p}{,}\PY{p}{(}\PY{n}{p}\PY{p}{,}\PY{l+m+mi}{0}\PY{p}{,}\PY{l+m+mi}{2}\PY{o}{*}\PY{n}{pi}\PY{p}{)}\PY{p}{)}

\PY{n+nb}{print}\PY{p}{(}\PY{l+s+s2}{\PYZdq{}}\PY{l+s+s2}{F = }\PY{l+s+si}{\PYZob{}:.4\PYZcb{}}\PY{l+s+s2}{az N}\PY{l+s+s2}{\PYZdq{}}\PY{o}{.}\PY{n}{format}\PY{p}{(}\PY{n}{F}\PY{o}{.}\PY{n}{evalf}\PY{p}{(}\PY{p}{)}\PY{p}{)}\PY{p}{)}
\end{Verbatim}
\end{tcolorbox}

    \begin{Verbatim}[commandchars=\\\{\}]
F = 16.56az N
    \end{Verbatim}

    \hypertarget{q2}{%
\subsection{Q2}\label{q2}}

    Para uma melhor compreensão do cálculo vetorial desenhou-se o exercicio
proposto:

    \begin{tcolorbox}[breakable, size=fbox, boxrule=1pt, pad at break*=1mm,colback=cellbackground, colframe=cellborder]
\prompt{In}{incolor}{24}{\boxspacing}
\begin{Verbatim}[commandchars=\\\{\}]
\PY{k+kn}{from} \PY{n+nn}{IPython}\PY{n+nn}{.}\PY{n+nn}{display} \PY{k+kn}{import} \PY{n}{Image}
\PY{n}{Image}\PY{p}{(}\PY{l+s+s2}{\PYZdq{}}\PY{l+s+s2}{lista 2 eletromag \PYZhy{} Q2.png}\PY{l+s+s2}{\PYZdq{}}\PY{p}{)}
\end{Verbatim}
\end{tcolorbox}
 
            
\prompt{Out}{outcolor}{24}{}
    
    \begin{center}
    \adjustimage{max size={0.9\linewidth}{0.9\paperheight}}{output_11_0.png}
    \end{center}
    { \hspace*{\fill} \\}
    

    

    pegou-se um ponto na linha de carga, cuja o vetor construído fosse
perpendicular a tal linha e na direção de \(B\) o ponto \((-2,-1,4)\),
para saber a direção do campo

    \begin{tcolorbox}[breakable, size=fbox, boxrule=1pt, pad at break*=1mm,colback=cellbackground, colframe=cellborder]
\prompt{In}{incolor}{25}{\boxspacing}
\begin{Verbatim}[commandchars=\\\{\}]
\PY{k+kn}{from} \PY{n+nn}{IPython}\PY{n+nn}{.}\PY{n+nn}{display} \PY{k+kn}{import} \PY{n}{Image}
\PY{n}{Image}\PY{p}{(}\PY{l+s+s2}{\PYZdq{}}\PY{l+s+s2}{lista 2 eletromag \PYZhy{} Q2(1).png}\PY{l+s+s2}{\PYZdq{}}\PY{p}{)}
\end{Verbatim}
\end{tcolorbox}
 
            
\prompt{Out}{outcolor}{25}{}
    
    \begin{center}
    \adjustimage{max size={0.9\linewidth}{0.9\paperheight}}{output_14_0.png}
    \end{center}
    { \hspace*{\fill} \\}
    

    posto isso pode-se definir o vetor: \[ \vec{R} = (-4a_x + 3a_y) \ V/m\]

    logo isso definido calculo-se o campo elétrico no ponto:
\[ \vec{E} = \frac{\rho_l}{2 \pi \ \epsilon_o \ R} \vec{a_R} \]

    \begin{tcolorbox}[breakable, size=fbox, boxrule=1pt, pad at break*=1mm,colback=cellbackground, colframe=cellborder]
\prompt{In}{incolor}{26}{\boxspacing}
\begin{Verbatim}[commandchars=\\\{\}]
\PY{k+kn}{import} \PY{n+nn}{numpy} \PY{k}{as} \PY{n+nn}{np} 
\PY{k+kn}{from} \PY{n+nn}{math} \PY{k+kn}{import} \PY{o}{*}

\PY{n}{rho2} \PY{o}{=} \PY{l+m+mf}{20e\PYZhy{}9}
\PY{n}{r} \PY{o}{=} \PY{n}{np}\PY{o}{.}\PY{n}{array}\PY{p}{(}\PY{p}{[}\PY{o}{\PYZhy{}}\PY{l+m+mi}{4}\PY{p}{,}\PY{l+m+mi}{3}\PY{p}{,}\PY{l+m+mi}{0}\PY{p}{]}\PY{p}{)}
\PY{n}{e0} \PY{o}{=} \PY{l+m+mf}{8.85e\PYZhy{}12}
\PY{n}{R} \PY{o}{=} \PY{n}{sqrt}\PY{p}{(}\PY{n+nb}{sum}\PY{p}{(}\PY{n}{r}\PY{o}{*}\PY{o}{*}\PY{l+m+mi}{2}\PY{p}{)}\PY{p}{)}

\PY{n}{ar} \PY{o}{=} \PY{n}{r}\PY{o}{/}\PY{n}{R}

\PY{n}{E} \PY{o}{=} \PY{p}{(}\PY{n}{rho}\PY{o}{/}\PY{p}{(}\PY{l+m+mi}{2}\PY{o}{*}\PY{n}{pi}\PY{o}{*}\PY{n}{e0}\PY{o}{*}\PY{n}{R}\PY{p}{)}\PY{p}{)}\PY{o}{*}\PY{n}{ar}

\PY{n+nb}{print}\PY{p}{(}\PY{l+s+s1}{\PYZsq{}}\PY{l+s+s1}{E = }\PY{l+s+si}{\PYZob{}:.4\PYZcb{}}\PY{l+s+s1}{*i }\PY{l+s+si}{\PYZob{}:.4\PYZcb{}}\PY{l+s+s1}{*j }\PY{l+s+s1}{\PYZsq{}}\PY{o}{.}\PY{n}{format}\PY{p}{(}\PY{o}{*}\PY{n}{E}\PY{p}{)}\PY{p}{)}
\end{Verbatim}
\end{tcolorbox}

    \begin{Verbatim}[commandchars=\\\{\}]
E = -5.755e+04*i 4.316e+04*j
    \end{Verbatim}

    \[ \vec{E} = (-57.55a_x + 43.2a_y) \ V/m\]

    \hypertarget{q3}{%
\subsection{Q3}\label{q3}}

    Primeiro adotou-se uma correção, de que o plano $(x=0)$ e $y=\pm y $,
posto isso pode-se desenhar a interseção dos planos e consequente
formação de 2 retas.

    \begin{tcolorbox}[breakable, size=fbox, boxrule=1pt, pad at break*=1mm,colback=cellbackground, colframe=cellborder]
\prompt{In}{incolor}{27}{\boxspacing}
\begin{Verbatim}[commandchars=\\\{\}]
\PY{k+kn}{from} \PY{n+nn}{IPython}\PY{n+nn}{.}\PY{n+nn}{display} \PY{k+kn}{import} \PY{n}{Image}
\PY{n}{Image}\PY{p}{(}\PY{l+s+s2}{\PYZdq{}}\PY{l+s+s2}{Q3.png}\PY{l+s+s2}{\PYZdq{}}\PY{p}{)}
\end{Verbatim}
\end{tcolorbox}
 
            
\prompt{Out}{outcolor}{27}{}
    
    \begin{center}
    \adjustimage{max size={0.9\linewidth}{0.9\paperheight}}{output_21_0.png}
    \end{center}
    { \hspace*{\fill} \\}
    

    Com isso, pode-se então ordenar pontos nessas retas (linhas de carga), e
assim formando vetores cujo são perpendiculares ao ponto \(B\), onde
este determina a intensidade do campo elétrico.

    Construiu-se o ponto \(A=(0,4,10)\) e \(B=(0,-4,10)\) ambos em suas
respectivas linhas de cargas, pode-se desenvolver o vetor
\(\vec{ab} = (4,-4,0)\) e \(\vec{cb} = (4,4,0)\) conseguindo assim
visualizar que o campo elétrico só vai estar na componente \(x\) uma vez
que a resultate presente é \(\vec{R} = (8,0,0)\)

    Aplicando a mesma relação de campo elétrico para distribuição de carga
para linhas infinitas tempos:
\[ \vec{E} = \frac{\rho_l}{2 \pi \ \epsilon_o \ R} \vec{a_R} \]

    \begin{tcolorbox}[breakable, size=fbox, boxrule=1pt, pad at break*=1mm,colback=cellbackground, colframe=cellborder]
\prompt{In}{incolor}{28}{\boxspacing}
\begin{Verbatim}[commandchars=\\\{\}]
\PY{n}{r} \PY{o}{=} \PY{n}{np}\PY{o}{.}\PY{n}{array}\PY{p}{(}\PY{p}{[}\PY{l+m+mi}{8}\PY{p}{,}\PY{l+m+mi}{0}\PY{p}{,}\PY{l+m+mi}{0}\PY{p}{]}\PY{p}{)}
\PY{n}{rho}\PY{o}{=}\PY{l+m+mf}{4e\PYZhy{}9}
\PY{n}{R} \PY{o}{=} \PY{n}{sqrt}\PY{p}{(}\PY{n+nb}{sum}\PY{p}{(}\PY{n}{r}\PY{p}{)}\PY{p}{)}
\PY{n}{e0} \PY{o}{=} \PY{l+m+mf}{8.85e\PYZhy{}12}

\PY{n}{ar} \PY{o}{=} \PY{n}{r}\PY{o}{/}\PY{n}{R}

\PY{n}{E} \PY{o}{=} \PY{p}{(}\PY{n}{rho}\PY{o}{/}\PY{p}{(}\PY{l+m+mi}{2}\PY{o}{*}\PY{n}{pi}\PY{o}{*}\PY{n}{e0}\PY{o}{*}\PY{n}{R}\PY{p}{)}\PY{p}{)}\PY{o}{*}\PY{n}{ar}

\PY{n+nb}{print}\PY{p}{(}\PY{l+s+s1}{\PYZsq{}}\PY{l+s+s1}{E = }\PY{l+s+si}{\PYZob{}:.4\PYZcb{}}\PY{l+s+s1}{ax}\PY{l+s+s1}{\PYZsq{}}\PY{o}{.}\PY{n}{format}\PY{p}{(}\PY{o}{*}\PY{n}{E}\PY{p}{)}\PY{p}{)}
\end{Verbatim}
\end{tcolorbox}

    \begin{Verbatim}[commandchars=\\\{\}]
E = 71.93ax
    \end{Verbatim}

    \[\vec{E} = 71.93 \ \vec{a}_x \ V/m\]

    

    \hypertarget{q4}{%
\subsection{Q4}\label{q4}}

    Com a simétria anulando os vetores em suas respectivas compenentes, o
campo elétrico no plano infinito independe da distância.
\[ \vec{E} = \frac{\rho_s}{2\epsilon_0} \vec{a}_z \]

    logo, para \(y>3\) tem-se só a alteração do sentido do vetor unitário
\(\vec{a_y}\) adota-se tal vetor como \[\vec{a}_y = (0,1,0)\]

    \begin{tcolorbox}[breakable, size=fbox, boxrule=1pt, pad at break*=1mm,colback=cellbackground, colframe=cellborder]
\prompt{In}{incolor}{29}{\boxspacing}
\begin{Verbatim}[commandchars=\\\{\}]
\PY{n}{rho} \PY{o}{=} \PY{l+m+mf}{1e\PYZhy{}8}\PY{o}{/}\PY{p}{(}\PY{l+m+mi}{6}\PY{o}{*}\PY{n}{pi}\PY{p}{)}
\PY{n}{E} \PY{o}{=} \PY{n}{rho}\PY{o}{/}\PY{p}{(}\PY{l+m+mi}{2}\PY{o}{*}\PY{n}{e0}\PY{p}{)}

\PY{n+nb}{print}\PY{p}{(}\PY{l+s+s1}{\PYZsq{}}\PY{l+s+s1}{E = }\PY{l+s+si}{\PYZob{}:.4\PYZcb{}}\PY{l+s+s1}{\PYZsq{}}\PY{o}{.}\PY{n}{format}\PY{p}{(}\PY{n}{E}\PY{p}{)}\PY{p}{)}
\end{Verbatim}
\end{tcolorbox}

    \begin{Verbatim}[commandchars=\\\{\}]
E = 29.97
    \end{Verbatim}

    

    $$ \vec{E} = 29.97 \ \vec{a}_y \ V/m $$

    para $(y<3$), isto é, $ \vec{a}\_y = (0,-1,0)$ o campo é

$$\vec{E} = -29.97 \ \vec{a}_y \ V/m $$

    \hypertarget{q5}{%
\subsection{Q5}\label{q5}}

    Os dois palnos infinitos de carga, fazem um campo elétrico resultante em
cada possível região do espaço, uam vez que a desindade dos mesmo é a
mesma.

    \begin{tcolorbox}[breakable, size=fbox, boxrule=1pt, pad at break*=1mm,colback=cellbackground, colframe=cellborder]
\prompt{In}{incolor}{30}{\boxspacing}
\begin{Verbatim}[commandchars=\\\{\}]
\PY{k+kn}{from} \PY{n+nn}{IPython}\PY{n+nn}{.}\PY{n+nn}{display} \PY{k+kn}{import} \PY{n}{Image}
\PY{n}{Image}\PY{p}{(}\PY{l+s+s2}{\PYZdq{}}\PY{l+s+s2}{Q5.png}\PY{l+s+s2}{\PYZdq{}}\PY{p}{)}
\end{Verbatim}
\end{tcolorbox}
 
            
\prompt{Out}{outcolor}{30}{}
    
    \begin{center}
    \adjustimage{max size={0.9\linewidth}{0.9\paperheight}}{output_37_0.png}
    \end{center}
    { \hspace*{\fill} \\}
    

$$
E_1 + E_2 =
\begin{cases}
\displaystyle -\left(\frac{\rho_s}{\epsilon_0}\right) \vec{a}_x  & \text{para \ x<-1}\\
\displaystyle 0                                  & \text{para \ -1<x<-1} \\
\displaystyle -\left(\frac{\rho_s}{\epsilon_0}\right) \vec{a}_x     & \text{para \ x>1}\\
\end{cases}
$$


    \hypertarget{q6}{%
\subsection{Q6}\label{q6}}

    \begin{tcolorbox}[breakable, size=fbox, boxrule=1pt, pad at break*=1mm,colback=cellbackground, colframe=cellborder]
\prompt{In}{incolor}{31}{\boxspacing}
\begin{Verbatim}[commandchars=\\\{\}]
\PY{k+kn}{from} \PY{n+nn}{IPython}\PY{n+nn}{.}\PY{n+nn}{display} \PY{k+kn}{import} \PY{n}{Image}
\PY{n}{Image}\PY{p}{(}\PY{l+s+s2}{\PYZdq{}}\PY{l+s+s2}{Q6.png}\PY{l+s+s2}{\PYZdq{}}\PY{p}{)}
\end{Verbatim}
\end{tcolorbox}
 
            
\prompt{Out}{outcolor}{31}{}
    
    \begin{center}
    \adjustimage{max size={0.9\linewidth}{0.9\paperheight}}{output_40_0.png}
    \end{center}
    { \hspace*{\fill} \\}
    

    Os vetores Azul e Verde apontam a direção do campo elétrico para nossa
reta \((x,-1,0)\). O vetor Verde corresponde ao campo saindo do plano
por tanto o vetor normal do campo em um plano
\[ \vec{E_s} = \frac{\rho_s}{2\epsilon_0} (-\vec{a}_z) \] Já o vetor
Azul corresponde ao campo saindo da linha de carga
\[ \vec{E}_L = \frac{\rho_l}{2 \pi \ \epsilon_o \ R} \vec{a_R} \] logo
sabe-se que o vetor resultate é a soma vetorial desses, para realizar
essa soma precisa-se do vetor que \(\vec{R}\) de \(\vec{a_R}\)

    Para construção do vetor pegou-se as duas retas
\(\vec{R} = (x,-1,0) - (x,3,-3)\) com isso
\[\vec{R} = (0,-4,3) = -4 \ \vec{a}_y + 3\ \vec{a}_z\] Como o campo tem
carga negativa, danda pela distribuição de densidade, tem que se adotar
que o vetor \(\vec{R}\) está sendo atraido pela linha de carga, logo o
que vai influênciar esse campo é um unitário de:
\[ \vec{E}_L = \frac{\rho_l}{2 \pi \ \epsilon_o \ R} (-\vec{a_R}) \]

    \begin{tcolorbox}[breakable, size=fbox, boxrule=1pt, pad at break*=1mm,colback=cellbackground, colframe=cellborder]
\prompt{In}{incolor}{32}{\boxspacing}
\begin{Verbatim}[commandchars=\\\{\}]
\PY{k+kn}{from} \PY{n+nn}{math} \PY{k+kn}{import} \PY{o}{*}

\PY{n}{rho\PYZus{}s} \PY{o}{=} \PY{l+m+mf}{1e\PYZhy{}9}\PY{o}{/}\PY{p}{(}\PY{l+m+mi}{3}\PY{o}{*}\PY{n}{pi}\PY{p}{)}
\PY{n}{rho\PYZus{}l} \PY{o}{=} \PY{l+m+mf}{25e\PYZhy{}9}\PY{o}{/}\PY{l+m+mi}{9} \PY{c+c1}{\PYZsh{}\PYZsh{}sinal da carga já foi interpratado no vetor unitário campo elétrico}

\PY{n}{R} \PY{o}{=} \PY{n}{np}\PY{o}{.}\PY{n}{array}\PY{p}{(}\PY{p}{[}\PY{l+m+mi}{0}\PY{p}{,}\PY{l+m+mi}{4}\PY{p}{,}\PY{o}{\PYZhy{}}\PY{l+m+mi}{3}\PY{p}{]}\PY{p}{)} \PY{c+c1}{\PYZsh{}sinal invertido para dar a referencia do campo }
\PY{n}{r} \PY{o}{=} \PY{n}{sqrt}\PY{p}{(}\PY{n+nb}{sum}\PY{p}{(}\PY{n}{R}\PY{o}{*}\PY{o}{*}\PY{l+m+mi}{2}\PY{p}{)}\PY{p}{)}

\PY{n}{ar} \PY{o}{=} \PY{n}{R}\PY{o}{/}\PY{n}{r}
\PY{n}{az} \PY{o}{=} \PY{n}{np}\PY{o}{.}\PY{n}{array}\PY{p}{(}\PY{p}{[}\PY{l+m+mi}{0}\PY{p}{,}\PY{l+m+mi}{0}\PY{p}{,}\PY{o}{\PYZhy{}}\PY{l+m+mi}{1}\PY{p}{]}\PY{p}{)}

\PY{n}{E\PYZus{}s} \PY{o}{=} \PY{n}{rho\PYZus{}s}\PY{o}{/}\PY{p}{(}\PY{l+m+mi}{2}\PY{o}{*}\PY{n}{e0}\PY{p}{)}\PY{o}{*}\PY{n}{az}
\PY{n}{E\PYZus{}l} \PY{o}{=} \PY{p}{(}\PY{n}{rho\PYZus{}l}\PY{o}{/}\PY{p}{(}\PY{l+m+mi}{2}\PY{o}{*}\PY{n}{pi}\PY{o}{*}\PY{n}{e0}\PY{o}{*}\PY{n}{r}\PY{p}{)}\PY{p}{)}\PY{o}{*}\PY{n}{ar}

\PY{n}{E\PYZus{}r} \PY{o}{=} \PY{n}{E\PYZus{}s} \PY{o}{+} \PY{n}{E\PYZus{}l}

\PY{n+nb}{print}\PY{p}{(}\PY{l+s+s2}{\PYZdq{}}\PY{l+s+s2}{E\PYZus{}r = }\PY{l+s+si}{\PYZob{}:.4\PYZcb{}}\PY{l+s+s2}{ax }\PY{l+s+si}{\PYZob{}:.4\PYZcb{}}\PY{l+s+s2}{ay }\PY{l+s+si}{\PYZob{}:.4\PYZcb{}}\PY{l+s+s2}{az}\PY{l+s+s2}{\PYZdq{}}\PY{o}{.}\PY{n}{format}\PY{p}{(}\PY{o}{*}\PY{n}{E\PYZus{}r}\PY{p}{)}\PY{p}{)}
\end{Verbatim}
\end{tcolorbox}

    \begin{Verbatim}[commandchars=\\\{\}]
E\_r = 0.0ax 7.993ay -11.99az
    \end{Verbatim}

    Então o vetor resultate é:
\[\vec{E}_r = 7.99 \ \vec{a}_y -11.99 \ \vec{a}_z  \ \ V/m\]

    \hypertarget{q7}{%
\subsection{Q7}\label{q7}}

    \begin{tcolorbox}[breakable, size=fbox, boxrule=1pt, pad at break*=1mm,colback=cellbackground, colframe=cellborder]
\prompt{In}{incolor}{33}{\boxspacing}
\begin{Verbatim}[commandchars=\\\{\}]
\PY{k+kn}{from} \PY{n+nn}{IPython}\PY{n+nn}{.}\PY{n+nn}{display} \PY{k+kn}{import} \PY{n}{Image}
\PY{n}{Image}\PY{p}{(}\PY{l+s+s2}{\PYZdq{}}\PY{l+s+s2}{Q7.png}\PY{l+s+s2}{\PYZdq{}}\PY{p}{)}
\end{Verbatim}
\end{tcolorbox}
 
            
\prompt{Out}{outcolor}{33}{}
    
    \begin{center}
    \adjustimage{max size={0.9\linewidth}{0.9\paperheight}}{output_46_0.png}
    \end{center}
    { \hspace*{\fill} \\}
    

    Pode-se identificar pelo desenho que o vetor \(u\) e \(v\) são vetores
normais dos planos propostos e que pode-se fácilmente se calcular
traçando um ponto em cada plano cuja a altura em \(z\) corresponda a
altura do ponto \(P\) o mesmo se faz na linha de carga. Posto todos os
vetores unitários no mesmo sentido podemos fazer a soma vetorial para
achar o campo resultade em \(P\).
\[E = \frac{\rho_l}{2 \pi \ \epsilon_o \ R} \vec{a_r} \ + \ \frac{\rho_{s_1}}{2 \ \epsilon_0 } \vec{a_{n_1}} \ + \ \frac{\rho_{s_2}}{2 \ \epsilon_o} \vec{a_{n_2}}\]

    \begin{tcolorbox}[breakable, size=fbox, boxrule=1pt, pad at break*=1mm,colback=cellbackground, colframe=cellborder]
\prompt{In}{incolor}{34}{\boxspacing}
\begin{Verbatim}[commandchars=\\\{\}]
\PY{c+c1}{\PYZsh{}\PYZsh{} Calculando vetor R }
\PY{n}{C} \PY{o}{=} \PY{n}{np}\PY{o}{.}\PY{n}{array}\PY{p}{(}\PY{p}{[}\PY{l+m+mi}{6}\PY{p}{,}\PY{l+m+mi}{0}\PY{p}{,}\PY{l+m+mi}{2}\PY{p}{]}\PY{p}{)} \PY{c+c1}{\PYZsh{}contido na linha de carga}
\PY{n}{P} \PY{o}{=} \PY{n}{np}\PY{o}{.}\PY{n}{array}\PY{p}{(}\PY{p}{[}\PY{l+m+mi}{2}\PY{p}{,}\PY{l+m+mi}{0}\PY{p}{,}\PY{l+m+mi}{2}\PY{p}{]}\PY{p}{)}

\PY{n}{r} \PY{o}{=} \PY{n}{P} \PY{o}{\PYZhy{}} \PY{n}{C} \PY{c+c1}{\PYZsh{}carga P é atraída para a distribuição no ponto em C }
\PY{n}{R} \PY{o}{=} \PY{n}{sqrt}\PY{p}{(}\PY{n+nb}{sum}\PY{p}{(}\PY{n}{r}\PY{o}{*}\PY{o}{*}\PY{l+m+mi}{2}\PY{p}{)}\PY{p}{)}
\PY{n}{ar} \PY{o}{=} \PY{n}{r}\PY{o}{/}\PY{n}{R}

\PY{c+c1}{\PYZsh{}\PYZsh{} calculando vetor an usando S1 que sera o vetor an de S1 com sentido diferente, como campo no plano não é influênciado pela distância(o modulo) então pode\PYZhy{}se simplismente pegar o unitário simétrico}

\PY{n}{B} \PY{o}{=} \PY{n}{np}\PY{o}{.}\PY{n}{array}\PY{p}{(}\PY{p}{[}\PY{l+m+mi}{0}\PY{p}{,}\PY{l+m+mi}{0}\PY{p}{,}\PY{l+m+mi}{2}\PY{p}{]}\PY{p}{)}
\PY{n}{n} \PY{o}{=} \PY{n}{P} \PY{o}{\PYZhy{}} \PY{n}{B} 
\PY{n}{N} \PY{o}{=} \PY{n}{sqrt}\PY{p}{(}\PY{n+nb}{sum}\PY{p}{(}\PY{n}{n}\PY{o}{*}\PY{o}{*}\PY{l+m+mi}{2}\PY{p}{)}\PY{p}{)}

\PY{n}{an1} \PY{o}{=} \PY{n}{n}\PY{o}{/}\PY{n}{N}

\PY{c+c1}{\PYZsh{}\PYZsh{}\PYZsh{}\PYZsh{}\PYZsh{}\PYZsh{}\PYZsh{}\PYZsh{}\PYZsh{}\PYZsh{}\PYZsh{}\PYZsh{}\PYZsh{}\PYZsh{}}
\PY{n}{C} \PY{o}{=} \PY{n}{np}\PY{o}{.}\PY{n}{array}\PY{p}{(}\PY{p}{[}\PY{l+m+mi}{6}\PY{p}{,}\PY{l+m+mi}{0}\PY{p}{,}\PY{l+m+mi}{2}\PY{p}{]}\PY{p}{)}
\PY{n}{n2} \PY{o}{=} \PY{n}{P}\PY{o}{\PYZhy{}}\PY{n}{C}
\PY{n}{N2} \PY{o}{=} \PY{n}{sqrt}\PY{p}{(}\PY{n+nb}{sum}\PY{p}{(}\PY{n}{n2}\PY{o}{*}\PY{o}{*}\PY{l+m+mi}{2}\PY{p}{)}\PY{p}{)}
\PY{n}{an2} \PY{o}{=} \PY{n}{n2}\PY{o}{/}\PY{n}{N2}


\PY{n}{rho\PYZus{}s} \PY{o}{=} \PY{l+m+mf}{1e\PYZhy{}9}\PY{o}{/}\PY{p}{(}\PY{l+m+mi}{3}\PY{o}{*}\PY{n}{pi}\PY{p}{)}
\PY{n}{rho\PYZus{}l} \PY{o}{=} \PY{o}{\PYZhy{}}\PY{l+m+mf}{2e\PYZhy{}9} \PY{c+c1}{\PYZsh{}\PYZsh{} vetor se mantem por que não interpretamos a carga no campo diretamente nos calculos vetoriais }


\PY{n}{E\PYZus{}s1} \PY{o}{=} \PY{n}{rho\PYZus{}s}\PY{o}{/}\PY{p}{(}\PY{l+m+mi}{2}\PY{o}{*}\PY{n}{e0}\PY{p}{)}\PY{o}{*}\PY{n}{an1}
\PY{n}{E\PYZus{}s2} \PY{o}{=} \PY{n}{rho\PYZus{}s}\PY{o}{/}\PY{p}{(}\PY{l+m+mi}{2}\PY{o}{*}\PY{n}{e0}\PY{p}{)}\PY{o}{*}\PY{p}{(}\PY{n}{an2}\PY{p}{)}

\PY{n}{E\PYZus{}l} \PY{o}{=} \PY{p}{(}\PY{n}{rho\PYZus{}l}\PY{o}{/}\PY{p}{(}\PY{l+m+mi}{2}\PY{o}{*}\PY{n}{pi}\PY{o}{*}\PY{n}{e0}\PY{o}{*}\PY{n}{R}\PY{p}{)}\PY{p}{)}\PY{o}{*}\PY{n}{ar}


\PY{n}{E} \PY{o}{=} \PY{n}{E\PYZus{}l} \PY{o}{+} \PY{n}{E\PYZus{}s1} \PY{o}{+} \PY{n}{E\PYZus{}s2} 


\PY{n+nb}{print}\PY{p}{(}\PY{l+s+s2}{\PYZdq{}}\PY{l+s+s2}{E = }\PY{l+s+si}{\PYZob{}:.4\PYZcb{}}\PY{l+s+s2}{ax }\PY{l+s+si}{\PYZob{}:.4\PYZcb{}}\PY{l+s+s2}{ay }\PY{l+s+si}{\PYZob{}:.4\PYZcb{}}\PY{l+s+s2}{az}\PY{l+s+s2}{\PYZdq{}}\PY{o}{.}\PY{n}{format}\PY{p}{(}\PY{o}{*}\PY{n}{E}\PY{p}{)}\PY{p}{)}
\end{Verbatim}
\end{tcolorbox}

    \begin{Verbatim}[commandchars=\\\{\}]
E = 8.992ax 0.0ay 0.0az
    \end{Verbatim}

    \[ E = 8.992 \ \vec{a}_x \ \ V/m \]

    \hypertarget{q8-e-q9-demostrativas}{%
\subsection{Q8 e Q9 Demostrativas}\label{q8-e-q9-demostrativas}}

    \hypertarget{q10}{%
\subsection{Q10}\label{q10}}

    Entende-se por distribuição senoidal que:
\[\rho_{(x)} = \theta_0 \ \sin{(\theta_1x)} \]

    onde \(\theta_0\) e \(\theta_1\) são constantes.

    vamos entender que a frequencia \(\theta_1\), de acordo com a figura,
seja \(\frac{\pi}{d}\) ficando
\[\rho_{(x)} = \theta_0 \ \sin{\left(\frac{\pi}{d} x\right)} \]

    sabe-se que \(Q = \int_L \rho_{(x)} \ dl\) desenvolvendo
\[Q = \int_0^x \theta_0 \ \sin{\left(\frac{\pi}{d} x\right)} \ dx\]

logo : $\theta\_1 = \frac{\pi Q}{d} $
$$\rho_L = \frac{\pi Q}{d} \sin{\left(\frac{\pi}{d}x\right)}$$

    \hypertarget{q11}{%
\subsection{Q11}\label{q11}}

    A densidade volumétrica é dada pelo pela razão de carga no volume:
\[\rho_v = \frac{Q}{V}\]

Como a questão não informou a densidade volumétrica pode-se descrever só
que o volume máximo é \[ v = \pi r^2 h \]

    \begin{tcolorbox}[breakable, size=fbox, boxrule=1pt, pad at break*=1mm,colback=cellbackground, colframe=cellborder]
\prompt{In}{incolor}{35}{\boxspacing}
\begin{Verbatim}[commandchars=\\\{\}]
\PY{n}{r} \PY{o}{=} \PY{l+m+mf}{0.1}
\PY{n}{h} \PY{o}{=} \PY{l+m+mf}{0.3}
\PY{n}{v} \PY{o}{=} \PY{n}{pi}\PY{o}{*}\PY{p}{(}\PY{n}{r}\PY{o}{*}\PY{o}{*}\PY{l+m+mi}{2}\PY{p}{)}\PY{o}{*}\PY{n}{h}
\PY{n+nb}{print}\PY{p}{(}\PY{l+s+s2}{\PYZdq{}}\PY{l+s+s2}{V = }\PY{l+s+si}{\PYZob{}:.2e\PYZcb{}}\PY{l+s+s2}{ m³}\PY{l+s+s2}{\PYZdq{}}\PY{o}{.}\PY{n}{format}\PY{p}{(}\PY{n}{v}\PY{p}{)}\PY{p}{)}
\end{Verbatim}
\end{tcolorbox}

    \begin{Verbatim}[commandchars=\\\{\}]
V = 9.42e-03 m³
    \end{Verbatim}

    logo pode-se expressar a máxima quantidade de carga nesse volume, em
função da distribuição de densidade \[Q = 9.42*10^{-3} \rho_v\]

    \hypertarget{q12}{%
\subsection{Q12}\label{q12}}

    Entendo que a distribuição de carga é dada por uma função seno, tem-se:
\[ \rho_{(x)} = A\sin{(Bx)}\]

e que a carga é dado por \[ Q = \iint \rho_{(x)} dxdy \]

De acordo com o gráfico sabe-se que a senoide passa por
\((\frac{1}{2},0)\) tendo em vista, os pares ordenados.


\begin{align*}
\rho_{(x)} &= A\sin{(Bx)} \\ \\
0 		   &= A \sin{\left(\frac{B}{2}\right)} \\ \\ 
B 		   &= 2*\arcsin{0} \\  \\
B 		   &= k\pi, \ \ \ \forall k \in \mathbf{Z}
\end{align*}


então, \[ \rho_{(x)} = A\sin{(k\pi x)}\]

integrando \(Q\) pode-se obter:


\begin{align*}
Q &= \iint \rho_{(x)} dxdy \\ \\ 
Q &= \int^1_0 \int^{0.5}_0 A\sin{(\pi x)} dxdy \\ \\ 
10p &=  A\int^1_0 \int^{0.5}_0 \sin{(\pi x)} dxdy \\ \\ 
A &=  \frac{10p}{ \int^1_0 \int^{0.5}_0 \sin{(\pi x)} dxdy}\\ \\ 
\end{align*}


    \begin{tcolorbox}[breakable, size=fbox, boxrule=1pt, pad at break*=1mm,colback=cellbackground, colframe=cellborder]
\prompt{In}{incolor}{36}{\boxspacing}
\begin{Verbatim}[commandchars=\\\{\}]
\PY{k+kn}{from} \PY{n+nn}{math} \PY{k+kn}{import} \PY{o}{*}
\PY{k+kn}{from} \PY{n+nn}{sympy} \PY{k+kn}{import} \PY{o}{*}

\PY{n}{x} \PY{o}{=} \PY{n}{Symbol}\PY{p}{(}\PY{l+s+s1}{\PYZsq{}}\PY{l+s+s1}{x}\PY{l+s+s1}{\PYZsq{}}\PY{p}{)}
\PY{n}{y} \PY{o}{=} \PY{n}{Symbol}\PY{p}{(}\PY{l+s+s1}{\PYZsq{}}\PY{l+s+s1}{y}\PY{l+s+s1}{\PYZsq{}}\PY{p}{)}
\PY{n}{f} \PY{o}{=} \PY{n}{sin}\PY{p}{(}\PY{n}{pi}\PY{o}{*}\PY{n}{x}\PY{p}{)}

 

\PY{n}{A} \PY{o}{=} \PY{l+m+mf}{10e\PYZhy{}12}\PY{o}{/}\PY{n}{integrate}\PY{p}{(}\PY{n}{integrate}\PY{p}{(}\PY{n}{f}\PY{p}{,}\PY{p}{(}\PY{n}{x}\PY{p}{,}\PY{l+m+mi}{0}\PY{p}{,}\PY{l+m+mf}{0.5}\PY{p}{)}\PY{p}{)}\PY{p}{,}\PY{p}{(}\PY{n}{y}\PY{p}{,}\PY{l+m+mi}{0}\PY{p}{,}\PY{l+m+mi}{1}\PY{p}{)}\PY{p}{)}

\PY{n}{A}
\end{Verbatim}
\end{tcolorbox}
 
            
\prompt{Out}{outcolor}{36}{}
    
    $\displaystyle 1.0 \cdot 10^{-11} \pi$

    

    \[ A = 0.1p \ \pi\]

    logo a expressão para densidade:
\[\rho_s = 0.1 \ \pi \sin{(k\pi \ x )} \ \ p C/m^2\] onde
\(k \in \mathbf{Z}\)

    \hypertarget{q13}{%
\subsection{Q13}\label{q13}}

    Semelhante a questão anteior, a senoide passa pelor par ordenado
\((45,0)\) logo caíra em um \(arcsen(0)\) que sugere pelo
desenvolvimento anterior que a frequência ângular dessa senoide seja um
\(k\frac{\pi}{45}\) para \(k\) sendo multiplo de \(45\), como mostra a
divisão em \(3\) partes pode-se adaptar a frequencia para

\[ \rho_{(x)} = A\sin{\left(\frac{k \pi}{15} x\right)}\]

tendo-se a carga de \(5nC\) integrou-se para assim acharmos \(A\)

    \begin{tcolorbox}[breakable, size=fbox, boxrule=1pt, pad at break*=1mm,colback=cellbackground, colframe=cellborder]
\prompt{In}{incolor}{37}{\boxspacing}
\begin{Verbatim}[commandchars=\\\{\}]
\PY{k+kn}{from} \PY{n+nn}{math} \PY{k+kn}{import} \PY{o}{*}
\PY{k+kn}{from} \PY{n+nn}{sympy} \PY{k+kn}{import} \PY{o}{*}

\PY{n}{x} \PY{o}{=} \PY{n}{Symbol}\PY{p}{(}\PY{l+s+s1}{\PYZsq{}}\PY{l+s+s1}{x}\PY{l+s+s1}{\PYZsq{}}\PY{p}{)}
\PY{n}{f} \PY{o}{=} \PY{n}{sin}\PY{p}{(}\PY{p}{(}\PY{n}{pi}\PY{o}{/}\PY{l+m+mi}{15}\PY{p}{)}\PY{o}{*}\PY{n}{x}\PY{p}{)}

\PY{n}{A} \PY{o}{=} \PY{l+m+mi}{5}\PY{o}{/}\PY{n}{integrate}\PY{p}{(}\PY{n}{f}\PY{p}{,}\PY{p}{(}\PY{n}{x}\PY{p}{,}\PY{l+m+mi}{0}\PY{p}{,}\PY{l+m+mi}{45}\PY{p}{)}\PY{p}{)}

\PY{n}{A}
\end{Verbatim}
\end{tcolorbox}
 
            
\prompt{Out}{outcolor}{37}{}
    
    $\displaystyle \frac{\pi}{6}$

    

    logo,
\[ \rho_{(x)} = \frac{\pi}{6}\sin{\left(\frac{k \pi}{15} x\right)} \ \ nC/m, \ \ \ \forall k \in M(15)\]


    % Add a bibliography block to the postdoc
    
    
    
\end{document}
