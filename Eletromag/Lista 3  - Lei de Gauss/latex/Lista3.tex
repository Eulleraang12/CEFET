\documentclass[11pt]{article}

    \usepackage[breakable]{tcolorbox}
    \usepackage{parskip} % Stop auto-indenting (to mimic markdown behaviour)
    \usepackage{amsmath}
    \usepackage{multirow}
    \usepackage{iftex}
    \ifPDFTeX
    	\usepackage[T1]{fontenc}
    	\usepackage{mathpazo}
    \else
    	\usepackage{fontspec}
    \fi

    % Basic figure setup, for now with no caption control since it's done
    % automatically by Pandoc (which extracts ![](path) syntax from Markdown).
    \usepackage{graphicx}
    % Maintain compatibility with old templates. Remove in nbconvert 6.0
    \let\Oldincludegraphics\includegraphics
    % Ensure that by default, figures have no caption (until we provide a
    % proper Figure object with a Caption API and a way to capture that
    % in the conversion process - todo).
    \usepackage{caption}
    \DeclareCaptionFormat{nocaption}{}
    \captionsetup{format=nocaption,aboveskip=0pt,belowskip=0pt}

    \usepackage{float}
    \floatplacement{figure}{H} % forces figures to be placed at the correct location
    \usepackage{xcolor} % Allow colors to be defined
    \usepackage{enumerate} % Needed for markdown enumerations to work
    \usepackage{geometry} % Used to adjust the document margins
    \usepackage{amsmath} % Equations
    \usepackage{amssymb} % Equations
    \usepackage{textcomp} % defines textquotesingle
    % Hack from http://tex.stackexchange.com/a/47451/13684:
    \AtBeginDocument{%
        \def\PYZsq{\textquotesingle}% Upright quotes in Pygmentized code
    }
    \usepackage{upquote} % Upright quotes for verbatim code
    \usepackage{eurosym} % defines \euro
    \usepackage[mathletters]{ucs} % Extended unicode (utf-8) support
    \usepackage{fancyvrb} % verbatim replacement that allows latex
    \usepackage{grffile} % extends the file name processing of package graphics 
                         % to support a larger range
    \makeatletter % fix for old versions of grffile with XeLaTeX
    \@ifpackagelater{grffile}{2019/11/01}
    {
      % Do nothing on new versions
    }
    {
      \def\Gread@@xetex#1{%
        \IfFileExists{"\Gin@base".bb}%
        {\Gread@eps{\Gin@base.bb}}%
        {\Gread@@xetex@aux#1}%
      }
    }
    \makeatother
    \usepackage[Export]{adjustbox} % Used to constrain images to a maximum size
    \adjustboxset{max size={0.9\linewidth}{0.9\paperheight}}

    % The hyperref package gives us a pdf with properly built
    % internal navigation ('pdf bookmarks' for the table of contents,
    % internal cross-reference links, web links for URLs, etc.)
    \usepackage{hyperref}
    % The default LaTeX title has an obnoxious amount of whitespace. By default,
    % titling removes some of it. It also provides customization options.
    \usepackage{titling}
    \usepackage{longtable} % longtable support required by pandoc >1.10
    \usepackage{booktabs}  % table support for pandoc > 1.12.2
    \usepackage[inline]{enumitem} % IRkernel/repr support (it uses the enumerate* environment)
    \usepackage[normalem]{ulem} % ulem is needed to support strikethroughs (\sout)
                                % normalem makes italics be italics, not underlines
    \usepackage{mathrsfs}
    

    
    % Colors for the hyperref package
    \definecolor{urlcolor}{rgb}{0,.145,.698}
    \definecolor{linkcolor}{rgb}{.71,0.21,0.01}
    \definecolor{citecolor}{rgb}{.12,.54,.11}

    % ANSI colors
    \definecolor{ansi-black}{HTML}{3E424D}
    \definecolor{ansi-black-intense}{HTML}{282C36}
    \definecolor{ansi-red}{HTML}{E75C58}
    \definecolor{ansi-red-intense}{HTML}{B22B31}
    \definecolor{ansi-green}{HTML}{00A250}
    \definecolor{ansi-green-intense}{HTML}{007427}
    \definecolor{ansi-yellow}{HTML}{DDB62B}
    \definecolor{ansi-yellow-intense}{HTML}{B27D12}
    \definecolor{ansi-blue}{HTML}{208FFB}
    \definecolor{ansi-blue-intense}{HTML}{0065CA}
    \definecolor{ansi-magenta}{HTML}{D160C4}
    \definecolor{ansi-magenta-intense}{HTML}{A03196}
    \definecolor{ansi-cyan}{HTML}{60C6C8}
    \definecolor{ansi-cyan-intense}{HTML}{258F8F}
    \definecolor{ansi-white}{HTML}{C5C1B4}
    \definecolor{ansi-white-intense}{HTML}{A1A6B2}
    \definecolor{ansi-default-inverse-fg}{HTML}{FFFFFF}
    \definecolor{ansi-default-inverse-bg}{HTML}{000000}

    % common color for the border for error outputs.
    \definecolor{outerrorbackground}{HTML}{FFDFDF}

    % commands and environments needed by pandoc snippets
    % extracted from the output of `pandoc -s`
    \providecommand{\tightlist}{%
      \setlength{\itemsep}{0pt}\setlength{\parskip}{0pt}}
    \DefineVerbatimEnvironment{Highlighting}{Verbatim}{commandchars=\\\{\}}
    % Add ',fontsize=\small' for more characters per line
    \newenvironment{Shaded}{}{}
    \newcommand{\KeywordTok}[1]{\textcolor[rgb]{0.00,0.44,0.13}{\textbf{{#1}}}}
    \newcommand{\DataTypeTok}[1]{\textcolor[rgb]{0.56,0.13,0.00}{{#1}}}
    \newcommand{\DecValTok}[1]{\textcolor[rgb]{0.25,0.63,0.44}{{#1}}}
    \newcommand{\BaseNTok}[1]{\textcolor[rgb]{0.25,0.63,0.44}{{#1}}}
    \newcommand{\FloatTok}[1]{\textcolor[rgb]{0.25,0.63,0.44}{{#1}}}
    \newcommand{\CharTok}[1]{\textcolor[rgb]{0.25,0.44,0.63}{{#1}}}
    \newcommand{\StringTok}[1]{\textcolor[rgb]{0.25,0.44,0.63}{{#1}}}
    \newcommand{\CommentTok}[1]{\textcolor[rgb]{0.38,0.63,0.69}{\textit{{#1}}}}
    \newcommand{\OtherTok}[1]{\textcolor[rgb]{0.00,0.44,0.13}{{#1}}}
    \newcommand{\AlertTok}[1]{\textcolor[rgb]{1.00,0.00,0.00}{\textbf{{#1}}}}
    \newcommand{\FunctionTok}[1]{\textcolor[rgb]{0.02,0.16,0.49}{{#1}}}
    \newcommand{\RegionMarkerTok}[1]{{#1}}
    \newcommand{\ErrorTok}[1]{\textcolor[rgb]{1.00,0.00,0.00}{\textbf{{#1}}}}
    \newcommand{\NormalTok}[1]{{#1}}
    
    % Additional commands for more recent versions of Pandoc
    \newcommand{\ConstantTok}[1]{\textcolor[rgb]{0.53,0.00,0.00}{{#1}}}
    \newcommand{\SpecialCharTok}[1]{\textcolor[rgb]{0.25,0.44,0.63}{{#1}}}
    \newcommand{\VerbatimStringTok}[1]{\textcolor[rgb]{0.25,0.44,0.63}{{#1}}}
    \newcommand{\SpecialStringTok}[1]{\textcolor[rgb]{0.73,0.40,0.53}{{#1}}}
    \newcommand{\ImportTok}[1]{{#1}}
    \newcommand{\DocumentationTok}[1]{\textcolor[rgb]{0.73,0.13,0.13}{\textit{{#1}}}}
    \newcommand{\AnnotationTok}[1]{\textcolor[rgb]{0.38,0.63,0.69}{\textbf{\textit{{#1}}}}}
    \newcommand{\CommentVarTok}[1]{\textcolor[rgb]{0.38,0.63,0.69}{\textbf{\textit{{#1}}}}}
    \newcommand{\VariableTok}[1]{\textcolor[rgb]{0.10,0.09,0.49}{{#1}}}
    \newcommand{\ControlFlowTok}[1]{\textcolor[rgb]{0.00,0.44,0.13}{\textbf{{#1}}}}
    \newcommand{\OperatorTok}[1]{\textcolor[rgb]{0.40,0.40,0.40}{{#1}}}
    \newcommand{\BuiltInTok}[1]{{#1}}
    \newcommand{\ExtensionTok}[1]{{#1}}
    \newcommand{\PreprocessorTok}[1]{\textcolor[rgb]{0.74,0.48,0.00}{{#1}}}
    \newcommand{\AttributeTok}[1]{\textcolor[rgb]{0.49,0.56,0.16}{{#1}}}
    \newcommand{\InformationTok}[1]{\textcolor[rgb]{0.38,0.63,0.69}{\textbf{\textit{{#1}}}}}
    \newcommand{\WarningTok}[1]{\textcolor[rgb]{0.38,0.63,0.69}{\textbf{\textit{{#1}}}}}
    
    
    % Define a nice break command that doesn't care if a line doesn't already
    % exist.
    \def\br{\hspace*{\fill} \\* }
    % Math Jax compatibility definitions
    \def\gt{>}
    \def\lt{<}
    \let\Oldtex\TeX
    \let\Oldlatex\LaTeX
    \renewcommand{\TeX}{\textrm{\Oldtex}}
    \renewcommand{\LaTeX}{\textrm{\Oldlatex}}
    % Document parameters
    % Document title
    \title{Lista3}
    
    
    
    
    
% Pygments definitions
\makeatletter
\def\PY@reset{\let\PY@it=\relax \let\PY@bf=\relax%
    \let\PY@ul=\relax \let\PY@tc=\relax%
    \let\PY@bc=\relax \let\PY@ff=\relax}
\def\PY@tok#1{\csname PY@tok@#1\endcsname}
\def\PY@toks#1+{\ifx\relax#1\empty\else%
    \PY@tok{#1}\expandafter\PY@toks\fi}
\def\PY@do#1{\PY@bc{\PY@tc{\PY@ul{%
    \PY@it{\PY@bf{\PY@ff{#1}}}}}}}
\def\PY#1#2{\PY@reset\PY@toks#1+\relax+\PY@do{#2}}

\expandafter\def\csname PY@tok@w\endcsname{\def\PY@tc##1{\textcolor[rgb]{0.73,0.73,0.73}{##1}}}
\expandafter\def\csname PY@tok@c\endcsname{\let\PY@it=\textit\def\PY@tc##1{\textcolor[rgb]{0.25,0.50,0.50}{##1}}}
\expandafter\def\csname PY@tok@cp\endcsname{\def\PY@tc##1{\textcolor[rgb]{0.74,0.48,0.00}{##1}}}
\expandafter\def\csname PY@tok@k\endcsname{\let\PY@bf=\textbf\def\PY@tc##1{\textcolor[rgb]{0.00,0.50,0.00}{##1}}}
\expandafter\def\csname PY@tok@kp\endcsname{\def\PY@tc##1{\textcolor[rgb]{0.00,0.50,0.00}{##1}}}
\expandafter\def\csname PY@tok@kt\endcsname{\def\PY@tc##1{\textcolor[rgb]{0.69,0.00,0.25}{##1}}}
\expandafter\def\csname PY@tok@o\endcsname{\def\PY@tc##1{\textcolor[rgb]{0.40,0.40,0.40}{##1}}}
\expandafter\def\csname PY@tok@ow\endcsname{\let\PY@bf=\textbf\def\PY@tc##1{\textcolor[rgb]{0.67,0.13,1.00}{##1}}}
\expandafter\def\csname PY@tok@nb\endcsname{\def\PY@tc##1{\textcolor[rgb]{0.00,0.50,0.00}{##1}}}
\expandafter\def\csname PY@tok@nf\endcsname{\def\PY@tc##1{\textcolor[rgb]{0.00,0.00,1.00}{##1}}}
\expandafter\def\csname PY@tok@nc\endcsname{\let\PY@bf=\textbf\def\PY@tc##1{\textcolor[rgb]{0.00,0.00,1.00}{##1}}}
\expandafter\def\csname PY@tok@nn\endcsname{\let\PY@bf=\textbf\def\PY@tc##1{\textcolor[rgb]{0.00,0.00,1.00}{##1}}}
\expandafter\def\csname PY@tok@ne\endcsname{\let\PY@bf=\textbf\def\PY@tc##1{\textcolor[rgb]{0.82,0.25,0.23}{##1}}}
\expandafter\def\csname PY@tok@nv\endcsname{\def\PY@tc##1{\textcolor[rgb]{0.10,0.09,0.49}{##1}}}
\expandafter\def\csname PY@tok@no\endcsname{\def\PY@tc##1{\textcolor[rgb]{0.53,0.00,0.00}{##1}}}
\expandafter\def\csname PY@tok@nl\endcsname{\def\PY@tc##1{\textcolor[rgb]{0.63,0.63,0.00}{##1}}}
\expandafter\def\csname PY@tok@ni\endcsname{\let\PY@bf=\textbf\def\PY@tc##1{\textcolor[rgb]{0.60,0.60,0.60}{##1}}}
\expandafter\def\csname PY@tok@na\endcsname{\def\PY@tc##1{\textcolor[rgb]{0.49,0.56,0.16}{##1}}}
\expandafter\def\csname PY@tok@nt\endcsname{\let\PY@bf=\textbf\def\PY@tc##1{\textcolor[rgb]{0.00,0.50,0.00}{##1}}}
\expandafter\def\csname PY@tok@nd\endcsname{\def\PY@tc##1{\textcolor[rgb]{0.67,0.13,1.00}{##1}}}
\expandafter\def\csname PY@tok@s\endcsname{\def\PY@tc##1{\textcolor[rgb]{0.73,0.13,0.13}{##1}}}
\expandafter\def\csname PY@tok@sd\endcsname{\let\PY@it=\textit\def\PY@tc##1{\textcolor[rgb]{0.73,0.13,0.13}{##1}}}
\expandafter\def\csname PY@tok@si\endcsname{\let\PY@bf=\textbf\def\PY@tc##1{\textcolor[rgb]{0.73,0.40,0.53}{##1}}}
\expandafter\def\csname PY@tok@se\endcsname{\let\PY@bf=\textbf\def\PY@tc##1{\textcolor[rgb]{0.73,0.40,0.13}{##1}}}
\expandafter\def\csname PY@tok@sr\endcsname{\def\PY@tc##1{\textcolor[rgb]{0.73,0.40,0.53}{##1}}}
\expandafter\def\csname PY@tok@ss\endcsname{\def\PY@tc##1{\textcolor[rgb]{0.10,0.09,0.49}{##1}}}
\expandafter\def\csname PY@tok@sx\endcsname{\def\PY@tc##1{\textcolor[rgb]{0.00,0.50,0.00}{##1}}}
\expandafter\def\csname PY@tok@m\endcsname{\def\PY@tc##1{\textcolor[rgb]{0.40,0.40,0.40}{##1}}}
\expandafter\def\csname PY@tok@gh\endcsname{\let\PY@bf=\textbf\def\PY@tc##1{\textcolor[rgb]{0.00,0.00,0.50}{##1}}}
\expandafter\def\csname PY@tok@gu\endcsname{\let\PY@bf=\textbf\def\PY@tc##1{\textcolor[rgb]{0.50,0.00,0.50}{##1}}}
\expandafter\def\csname PY@tok@gd\endcsname{\def\PY@tc##1{\textcolor[rgb]{0.63,0.00,0.00}{##1}}}
\expandafter\def\csname PY@tok@gi\endcsname{\def\PY@tc##1{\textcolor[rgb]{0.00,0.63,0.00}{##1}}}
\expandafter\def\csname PY@tok@gr\endcsname{\def\PY@tc##1{\textcolor[rgb]{1.00,0.00,0.00}{##1}}}
\expandafter\def\csname PY@tok@ge\endcsname{\let\PY@it=\textit}
\expandafter\def\csname PY@tok@gs\endcsname{\let\PY@bf=\textbf}
\expandafter\def\csname PY@tok@gp\endcsname{\let\PY@bf=\textbf\def\PY@tc##1{\textcolor[rgb]{0.00,0.00,0.50}{##1}}}
\expandafter\def\csname PY@tok@go\endcsname{\def\PY@tc##1{\textcolor[rgb]{0.53,0.53,0.53}{##1}}}
\expandafter\def\csname PY@tok@gt\endcsname{\def\PY@tc##1{\textcolor[rgb]{0.00,0.27,0.87}{##1}}}
\expandafter\def\csname PY@tok@err\endcsname{\def\PY@bc##1{\setlength{\fboxsep}{0pt}\fcolorbox[rgb]{1.00,0.00,0.00}{1,1,1}{\strut ##1}}}
\expandafter\def\csname PY@tok@kc\endcsname{\let\PY@bf=\textbf\def\PY@tc##1{\textcolor[rgb]{0.00,0.50,0.00}{##1}}}
\expandafter\def\csname PY@tok@kd\endcsname{\let\PY@bf=\textbf\def\PY@tc##1{\textcolor[rgb]{0.00,0.50,0.00}{##1}}}
\expandafter\def\csname PY@tok@kn\endcsname{\let\PY@bf=\textbf\def\PY@tc##1{\textcolor[rgb]{0.00,0.50,0.00}{##1}}}
\expandafter\def\csname PY@tok@kr\endcsname{\let\PY@bf=\textbf\def\PY@tc##1{\textcolor[rgb]{0.00,0.50,0.00}{##1}}}
\expandafter\def\csname PY@tok@bp\endcsname{\def\PY@tc##1{\textcolor[rgb]{0.00,0.50,0.00}{##1}}}
\expandafter\def\csname PY@tok@fm\endcsname{\def\PY@tc##1{\textcolor[rgb]{0.00,0.00,1.00}{##1}}}
\expandafter\def\csname PY@tok@vc\endcsname{\def\PY@tc##1{\textcolor[rgb]{0.10,0.09,0.49}{##1}}}
\expandafter\def\csname PY@tok@vg\endcsname{\def\PY@tc##1{\textcolor[rgb]{0.10,0.09,0.49}{##1}}}
\expandafter\def\csname PY@tok@vi\endcsname{\def\PY@tc##1{\textcolor[rgb]{0.10,0.09,0.49}{##1}}}
\expandafter\def\csname PY@tok@vm\endcsname{\def\PY@tc##1{\textcolor[rgb]{0.10,0.09,0.49}{##1}}}
\expandafter\def\csname PY@tok@sa\endcsname{\def\PY@tc##1{\textcolor[rgb]{0.73,0.13,0.13}{##1}}}
\expandafter\def\csname PY@tok@sb\endcsname{\def\PY@tc##1{\textcolor[rgb]{0.73,0.13,0.13}{##1}}}
\expandafter\def\csname PY@tok@sc\endcsname{\def\PY@tc##1{\textcolor[rgb]{0.73,0.13,0.13}{##1}}}
\expandafter\def\csname PY@tok@dl\endcsname{\def\PY@tc##1{\textcolor[rgb]{0.73,0.13,0.13}{##1}}}
\expandafter\def\csname PY@tok@s2\endcsname{\def\PY@tc##1{\textcolor[rgb]{0.73,0.13,0.13}{##1}}}
\expandafter\def\csname PY@tok@sh\endcsname{\def\PY@tc##1{\textcolor[rgb]{0.73,0.13,0.13}{##1}}}
\expandafter\def\csname PY@tok@s1\endcsname{\def\PY@tc##1{\textcolor[rgb]{0.73,0.13,0.13}{##1}}}
\expandafter\def\csname PY@tok@mb\endcsname{\def\PY@tc##1{\textcolor[rgb]{0.40,0.40,0.40}{##1}}}
\expandafter\def\csname PY@tok@mf\endcsname{\def\PY@tc##1{\textcolor[rgb]{0.40,0.40,0.40}{##1}}}
\expandafter\def\csname PY@tok@mh\endcsname{\def\PY@tc##1{\textcolor[rgb]{0.40,0.40,0.40}{##1}}}
\expandafter\def\csname PY@tok@mi\endcsname{\def\PY@tc##1{\textcolor[rgb]{0.40,0.40,0.40}{##1}}}
\expandafter\def\csname PY@tok@il\endcsname{\def\PY@tc##1{\textcolor[rgb]{0.40,0.40,0.40}{##1}}}
\expandafter\def\csname PY@tok@mo\endcsname{\def\PY@tc##1{\textcolor[rgb]{0.40,0.40,0.40}{##1}}}
\expandafter\def\csname PY@tok@ch\endcsname{\let\PY@it=\textit\def\PY@tc##1{\textcolor[rgb]{0.25,0.50,0.50}{##1}}}
\expandafter\def\csname PY@tok@cm\endcsname{\let\PY@it=\textit\def\PY@tc##1{\textcolor[rgb]{0.25,0.50,0.50}{##1}}}
\expandafter\def\csname PY@tok@cpf\endcsname{\let\PY@it=\textit\def\PY@tc##1{\textcolor[rgb]{0.25,0.50,0.50}{##1}}}
\expandafter\def\csname PY@tok@c1\endcsname{\let\PY@it=\textit\def\PY@tc##1{\textcolor[rgb]{0.25,0.50,0.50}{##1}}}
\expandafter\def\csname PY@tok@cs\endcsname{\let\PY@it=\textit\def\PY@tc##1{\textcolor[rgb]{0.25,0.50,0.50}{##1}}}

\def\PYZbs{\char`\\}
\def\PYZus{\char`\_}
\def\PYZob{\char`\{}
\def\PYZcb{\char`\}}
\def\PYZca{\char`\^}
\def\PYZam{\char`\&}
\def\PYZlt{\char`\<}
\def\PYZgt{\char`\>}
\def\PYZsh{\char`\#}
\def\PYZpc{\char`\%}
\def\PYZdl{\char`\$}
\def\PYZhy{\char`\-}
\def\PYZsq{\char`\'}
\def\PYZdq{\char`\"}
\def\PYZti{\char`\~}
% for compatibility with earlier versions
\def\PYZat{@}
\def\PYZlb{[}
\def\PYZrb{]}
\makeatother


    % For linebreaks inside Verbatim environment from package fancyvrb. 
    \makeatletter
        \newbox\Wrappedcontinuationbox 
        \newbox\Wrappedvisiblespacebox 
        \newcommand*\Wrappedvisiblespace {\textcolor{red}{\textvisiblespace}} 
        \newcommand*\Wrappedcontinuationsymbol {\textcolor{red}{\llap{\tiny$\m@th\hookrightarrow$}}} 
        \newcommand*\Wrappedcontinuationindent {3ex } 
        \newcommand*\Wrappedafterbreak {\kern\Wrappedcontinuationindent\copy\Wrappedcontinuationbox} 
        % Take advantage of the already applied Pygments mark-up to insert 
        % potential linebreaks for TeX processing. 
        %        {, <, #, %, $, ' and ": go to next line. 
        %        _, }, ^, &, >, - and ~: stay at end of broken line. 
        % Use of \textquotesingle for straight quote. 
        \newcommand*\Wrappedbreaksatspecials {% 
            \def\PYGZus{\discretionary{\char`\_}{\Wrappedafterbreak}{\char`\_}}% 
            \def\PYGZob{\discretionary{}{\Wrappedafterbreak\char`\{}{\char`\{}}% 
            \def\PYGZcb{\discretionary{\char`\}}{\Wrappedafterbreak}{\char`\}}}% 
            \def\PYGZca{\discretionary{\char`\^}{\Wrappedafterbreak}{\char`\^}}% 
            \def\PYGZam{\discretionary{\char`\&}{\Wrappedafterbreak}{\char`\&}}% 
            \def\PYGZlt{\discretionary{}{\Wrappedafterbreak\char`\<}{\char`\<}}% 
            \def\PYGZgt{\discretionary{\char`\>}{\Wrappedafterbreak}{\char`\>}}% 
            \def\PYGZsh{\discretionary{}{\Wrappedafterbreak\char`\#}{\char`\#}}% 
            \def\PYGZpc{\discretionary{}{\Wrappedafterbreak\char`\%}{\char`\%}}% 
            \def\PYGZdl{\discretionary{}{\Wrappedafterbreak\char`\$}{\char`\$}}% 
            \def\PYGZhy{\discretionary{\char`\-}{\Wrappedafterbreak}{\char`\-}}% 
            \def\PYGZsq{\discretionary{}{\Wrappedafterbreak\textquotesingle}{\textquotesingle}}% 
            \def\PYGZdq{\discretionary{}{\Wrappedafterbreak\char`\"}{\char`\"}}% 
            \def\PYGZti{\discretionary{\char`\~}{\Wrappedafterbreak}{\char`\~}}% 
        } 
        % Some characters . , ; ? ! / are not pygmentized. 
        % This macro makes them "active" and they will insert potential linebreaks 
        \newcommand*\Wrappedbreaksatpunct {% 
            \lccode`\~`\.\lowercase{\def~}{\discretionary{\hbox{\char`\.}}{\Wrappedafterbreak}{\hbox{\char`\.}}}% 
            \lccode`\~`\,\lowercase{\def~}{\discretionary{\hbox{\char`\,}}{\Wrappedafterbreak}{\hbox{\char`\,}}}% 
            \lccode`\~`\;\lowercase{\def~}{\discretionary{\hbox{\char`\;}}{\Wrappedafterbreak}{\hbox{\char`\;}}}% 
            \lccode`\~`\:\lowercase{\def~}{\discretionary{\hbox{\char`\:}}{\Wrappedafterbreak}{\hbox{\char`\:}}}% 
            \lccode`\~`\?\lowercase{\def~}{\discretionary{\hbox{\char`\?}}{\Wrappedafterbreak}{\hbox{\char`\?}}}% 
            \lccode`\~`\!\lowercase{\def~}{\discretionary{\hbox{\char`\!}}{\Wrappedafterbreak}{\hbox{\char`\!}}}% 
            \lccode`\~`\/\lowercase{\def~}{\discretionary{\hbox{\char`\/}}{\Wrappedafterbreak}{\hbox{\char`\/}}}% 
            \catcode`\.\active
            \catcode`\,\active 
            \catcode`\;\active
            \catcode`\:\active
            \catcode`\?\active
            \catcode`\!\active
            \catcode`\/\active 
            \lccode`\~`\~ 	
        }
    \makeatother

    \let\OriginalVerbatim=\Verbatim
    \makeatletter
    \renewcommand{\Verbatim}[1][1]{%
        %\parskip\z@skip
        \sbox\Wrappedcontinuationbox {\Wrappedcontinuationsymbol}%
        \sbox\Wrappedvisiblespacebox {\FV@SetupFont\Wrappedvisiblespace}%
        \def\FancyVerbFormatLine ##1{\hsize\linewidth
            \vtop{\raggedright\hyphenpenalty\z@\exhyphenpenalty\z@
                \doublehyphendemerits\z@\finalhyphendemerits\z@
                \strut ##1\strut}%
        }%
        % If the linebreak is at a space, the latter will be displayed as visible
        % space at end of first line, and a continuation symbol starts next line.
        % Stretch/shrink are however usually zero for typewriter font.
        \def\FV@Space {%
            \nobreak\hskip\z@ plus\fontdimen3\font minus\fontdimen4\font
            \discretionary{\copy\Wrappedvisiblespacebox}{\Wrappedafterbreak}
            {\kern\fontdimen2\font}%
        }%
        
        % Allow breaks at special characters using \PYG... macros.
        \Wrappedbreaksatspecials
        % Breaks at punctuation characters . , ; ? ! and / need catcode=\active 	
        \OriginalVerbatim[#1,codes*=\Wrappedbreaksatpunct]%
    }
    \makeatother

    % Exact colors from NB
    \definecolor{incolor}{HTML}{303F9F}
    \definecolor{outcolor}{HTML}{D84315}
    \definecolor{cellborder}{HTML}{CFCFCF}
    \definecolor{cellbackground}{HTML}{F7F7F7}
    
    % prompt
    \makeatletter
    \newcommand{\boxspacing}{\kern\kvtcb@left@rule\kern\kvtcb@boxsep}
    \makeatother
    \newcommand{\prompt}[4]{
        {\ttfamily\llap{{\color{#2}[#3]:\hspace{3pt}#4}}\vspace{-\baselineskip}}
    }
    

    
    % Prevent overflowing lines due to hard-to-break entities
    \sloppy 
    % Setup hyperref package
    \hypersetup{
      breaklinks=true,  % so long urls are correctly broken across lines
      colorlinks=true,
      urlcolor=urlcolor,
      linkcolor=linkcolor,
      citecolor=citecolor,
      }
    % Slightly bigger margins than the latex defaults
    
    \geometry{verbose,tmargin=1in,bmargin=1in,lmargin=1in,rmargin=1in}
    
    

\begin{document}
    
    \maketitle
    
    

    
    \hypertarget{lista-3}{%
\section{Lista 3}\label{lista-3}}

\hypertarget{q1}{%
\subsection{Q1}\label{q1}}

    traçando um cilindro para manter a simétria com a linha infinita,
podemos descrever que a carga interna da gaussiana é dada por:
\[ q_{int} = \int \rho_L dz \] \[ q_{int} = \rho_L z \]

já na lei de gauss
\[ \oint \vec{E} \vec{n} dA = \frac{q_{int}}{\epsilon_0} \]

sabendo que a area do cilindro é $A = 2\pi r ~z $ podemos concluir que
$$ EA = \frac{\rho_L z}{\epsilon_0} $$

logo \[ E = \frac{\rho_L }{2\pi\epsilon_0 r } \]

uma vez que o fluxo elétrico é definido por
\(\vec{D} = \epsilon_0 \vec{E}\) \[ D = \frac{\rho_L}{2\pi r}\]

    \hypertarget{q2}{%
\section{Q2}\label{q2}}

    relembrando o teorema da divergência
\[\iint_S \vec{D} \cdot \vec{n} dS = \iiint_V \vec{\nabla} \cdot \vec{D} dV \]

    \[ \vec{\nabla} \cdot \vec{D} = \frac{\partial D_x}{\partial x}+\frac{\partial D_y}{\partial y}+\frac{\partial D_z}{\partial z}\]
logo desenvolvendo

$$    \begin{cases}
\displaystyle \frac{\partial D_x}{\partial x} &= \displaystyle \frac{\partial yz\cos{y^2}}{\partial x} = 0 \\ \\ 
\displaystyle\frac{\partial D_y}{\partial y} &= \displaystyle\frac{\partial xz\cos{x^2}}{\partial y} = 0 \\ \\ 
\displaystyle\frac{\partial D_z}{\partial z} &= \displaystyle\frac{\partial 1}{\partial z} = 0
\end{cases}$$

    como \(\vec{\nabla} \cdot \vec{D} = 0\) diz que temos um fluxo nulo, a
quantidade que entra é igual ao que sai a corrente é constante.

    \hypertarget{q3}{%
\subsection{Q3}\label{q3}}

    Calculando face a face do cubo pode-se observar que para

$$    \begin{cases}
\\
x=0 &\rightarrow \displaystyle\int_{atras} =0 \\
x=2 &\rightarrow \displaystyle\int_{frente} =\int_0^2\int_0^2 xy^2 (ax \cdot ax) dydz = \frac{32}{3}\\
y=0 &\rightarrow \displaystyle\int_{esquerda} =0 \\
y=2 &\rightarrow \displaystyle\int_{direita} = \int_0^2\int_0^2 yx^2 (ay \cdot ay) dxdz = \frac{32}{3}\\
\end{cases}$$

logo a soma \[\oint Dds = \frac{64}{3} \ C\]

    já pelo outro lado
\[ \nabla D = \frac{\partial xy^2}{\partial x} + \frac{\partial yx^2}{\partial y}\]

    Resolvendo a integral:

\[ \int \nabla D dv = \int_0^2\int_0^2\int_0^2 (x^2+y^2) dxdydz \]

\[ \int \nabla D dv = \frac{64}{3} \]

    \hypertarget{q4}{%
\section{Q4}\label{q4}}

    Calculando cada face do cubo

$$\begin{cases}
\\
x=-1 &\rightarrow \displaystyle\int_{atras} =\int_{-1}^1\int_{-1}^1 10x^2 (ax \cdot ax) dydz = 40\\ \\
x=1 &\rightarrow \displaystyle\int_{frente} =\int_{-1}^1\int_{-1}^1 10x^2 (ax \cdot (-ax)) dydz = -40\\
y=-1 &\rightarrow \displaystyle\int_{esquerda} =0 \\
y=1 &\rightarrow \displaystyle\int_{direita} = 0\\
\end{cases}$$

Logo pode-se ver que o fluxo é nulo

já do outro lado do teorema
\[ \nabla D = \frac{\partial 10x^2}{\partial x} + \frac{\partial xy^3}{\partial y} = 20x+3xy^2\]

\[ \int \nabla D dv = \int_{-1}^1\int_{-1}^1\int_{-1}^1 20x+3xy^2 dxdydz = 0\]

    \begin{tcolorbox}[breakable, size=fbox, boxrule=1pt, pad at break*=1mm,colback=cellbackground, colframe=cellborder]
\prompt{In}{incolor}{1}{\boxspacing}
\begin{Verbatim}[commandchars=\\\{\}]
\PY{k+kn}{from} \PY{n+nn}{math} \PY{k+kn}{import} \PY{o}{*}
\PY{k+kn}{import} \PY{n+nn}{numpy} \PY{k}{as} \PY{n+nn}{np}
\PY{k+kn}{import} \PY{n+nn}{sympy} \PY{k}{as} \PY{n+nn}{sp}

\PY{n}{x} \PY{o}{=} \PY{n}{sp}\PY{o}{.}\PY{n}{Symbol}\PY{p}{(}\PY{l+s+s1}{\PYZsq{}}\PY{l+s+s1}{x}\PY{l+s+s1}{\PYZsq{}}\PY{p}{)}
\PY{n}{y} \PY{o}{=} \PY{n}{sp}\PY{o}{.}\PY{n}{Symbol}\PY{p}{(}\PY{l+s+s1}{\PYZsq{}}\PY{l+s+s1}{y}\PY{l+s+s1}{\PYZsq{}}\PY{p}{)}
\PY{n}{z} \PY{o}{=} \PY{n}{sp}\PY{o}{.}\PY{n}{Symbol}\PY{p}{(}\PY{l+s+s1}{\PYZsq{}}\PY{l+s+s1}{z}\PY{l+s+s1}{\PYZsq{}}\PY{p}{)}

\PY{n}{f} \PY{o}{=} \PY{l+m+mi}{20}\PY{o}{*}\PY{n}{x}\PY{o}{+}\PY{l+m+mi}{3}\PY{o}{*}\PY{n}{x}\PY{o}{*}\PY{n}{y}\PY{o}{*}\PY{o}{*}\PY{l+m+mi}{2}

\PY{n}{Dv} \PY{o}{=} \PY{n}{sp}\PY{o}{.}\PY{n}{integrate}\PY{p}{(}\PY{n}{sp}\PY{o}{.}\PY{n}{integrate}\PY{p}{(}\PY{n}{sp}\PY{o}{.}\PY{n}{integrate}\PY{p}{(}\PY{n}{f}\PY{p}{,}\PY{p}{(}\PY{n}{x}\PY{p}{,}\PY{o}{\PYZhy{}}\PY{l+m+mi}{1}\PY{p}{,}\PY{l+m+mi}{1}\PY{p}{)}\PY{p}{)}\PY{p}{,}\PY{p}{(}\PY{n}{y}\PY{p}{,}\PY{o}{\PYZhy{}}\PY{l+m+mi}{1}\PY{p}{,}\PY{l+m+mi}{1}\PY{p}{)}\PY{p}{)}\PY{p}{,}\PY{p}{(}\PY{n}{z}\PY{p}{,}\PY{o}{\PYZhy{}}\PY{l+m+mi}{1}\PY{p}{,}\PY{l+m+mi}{1}\PY{p}{)}\PY{p}{)}
\PY{n}{Dv}
\end{Verbatim}
\end{tcolorbox}
 
            
\prompt{Out}{outcolor}{1}{}
    
    $$\displaystyle 0$$

    

    

    \hypertarget{q5}{%
\section{Q5}\label{q5}}

    Sabendo que \[ \rho_v = \vec{\nabla} \vec{D}\]

    \hypertarget{a}{%
\subsubsection{a)}\label{a}}

    \[ \rho_v = \frac{1}{\rho} \frac{\partial (\rho D_{\rho})}{\partial \rho}\]

    \begin{tcolorbox}[breakable, size=fbox, boxrule=1pt, pad at break*=1mm,colback=cellbackground, colframe=cellborder]
\prompt{In}{incolor}{2}{\boxspacing}
\begin{Verbatim}[commandchars=\\\{\}]
\PY{n}{p} \PY{o}{=} \PY{n}{sp}\PY{o}{.}\PY{n}{Symbol}\PY{p}{(}\PY{l+s+s1}{\PYZsq{}}\PY{l+s+s1}{p}\PY{l+s+s1}{\PYZsq{}}\PY{p}{)}    
   
\PY{n}{Dp} \PY{o}{=} \PY{l+m+mi}{53}\PY{o}{*}\PY{n}{p}\PY{o}{*}\PY{o}{*}\PY{l+m+mi}{3}
\PY{n}{pv} \PY{o}{=} \PY{p}{(}\PY{l+m+mi}{1}\PY{o}{/}\PY{n}{p}\PY{p}{)}\PY{o}{*}\PY{n}{sp}\PY{o}{.}\PY{n}{diff}\PY{p}{(}\PY{n}{p}\PY{o}{*}\PY{n}{Dp}\PY{p}{,}\PY{n}{p}\PY{p}{)}

\PY{n}{pv}\PY{o}{.}\PY{n}{subs}\PY{p}{(}\PY{n}{p}\PY{p}{,}\PY{l+m+mi}{3}\PY{p}{)} 
\end{Verbatim}
\end{tcolorbox}
 
            
\prompt{Out}{outcolor}{2}{}
    
    $$\displaystyle 1908$$

    

    \[ \rho_v = 1908 \ C/m^3\]

    \hypertarget{b}{%
\subsubsection{b)}\label{b}}

    \begin{tcolorbox}[breakable, size=fbox, boxrule=1pt, pad at break*=1mm,colback=cellbackground, colframe=cellborder]
\prompt{In}{incolor}{3}{\boxspacing}
\begin{Verbatim}[commandchars=\\\{\}]
\PY{n}{Dp}\PY{o}{.}\PY{n}{subs}\PY{p}{(}\PY{n}{p}\PY{p}{,}\PY{l+m+mi}{3}\PY{p}{)}
\end{Verbatim}
\end{tcolorbox}
 
            
\prompt{Out}{outcolor}{3}{}
    
    $$\displaystyle 1431$$

    

    \[ \vec{D} = 1431 \vec{a}_{\rho} \ C/m^3\]

    \hypertarget{c}{%
\subsection{c)}\label{c}}

    \begin{align*}
 \psi &= \oint \vec{D}dA \\
 \psi &= \int_{-2.5}^{2.5}\int_{-2.5}^{2.5} 53\rho^3 \ \rho d\phi dz 
 \end{align*}

    \begin{tcolorbox}[breakable, size=fbox, boxrule=1pt, pad at break*=1mm,colback=cellbackground, colframe=cellborder]
\prompt{In}{incolor}{4}{\boxspacing}
\begin{Verbatim}[commandchars=\\\{\}]
\PY{n}{phi} \PY{o}{=} \PY{n}{sp}\PY{o}{.}\PY{n}{Symbol}\PY{p}{(}\PY{l+s+s2}{\PYZdq{}}\PY{l+s+s2}{phi}\PY{l+s+s2}{\PYZdq{}}\PY{p}{)}
\PY{n}{Dv} \PY{o}{=} \PY{l+m+mi}{53}\PY{o}{*}\PY{n}{p}\PY{o}{*}\PY{o}{*}\PY{l+m+mi}{3}

\PY{n}{psi} \PY{o}{=} \PY{n}{sp}\PY{o}{.}\PY{n}{integrate}\PY{p}{(}\PY{n}{sp}\PY{o}{.}\PY{n}{integrate}\PY{p}{(}\PY{n}{Dv}\PY{o}{*}\PY{n}{p}\PY{p}{,}\PY{p}{(}\PY{n}{p}\PY{p}{,}\PY{o}{\PYZhy{}}\PY{l+m+mf}{2.5}\PY{p}{,}\PY{l+m+mf}{2.5}\PY{p}{)}\PY{p}{)}\PY{p}{,}\PY{p}{(}\PY{n}{phi}\PY{p}{,}\PY{l+m+mi}{0}\PY{p}{,}\PY{l+m+mi}{2}\PY{o}{*}\PY{n}{pi}\PY{p}{)}\PY{p}{)}
\PY{n}{psi}
\end{Verbatim}
\end{tcolorbox}
 
            
\prompt{Out}{outcolor}{4}{}
    
    $$\displaystyle 13008.1570812702$$

    

    \[ \psi = 13008.157 \ C \]

    \hypertarget{d}{%
\subsubsection{d)}\label{d}}

    \[ Q_{int} = \psi = 13008.157 \ C\]

    \hypertarget{q6}{%
\subsection{Q6}\label{q6}}

    Sabendo que \[ Q_{int} = \oint \vec{\nabla}\cdot\vec{D} dv \]

primeiro calculo do divergente

\[ \vec{\nabla}\cdot\vec{D} = \frac{1}{\rho} \frac{\partial (\rho D_{\rho})}{\partial \rho}+ \frac{1}{\rho} \frac{\partial (D_{\phi})}{\partial \phi} + \frac{\partial (D_{z})}{\partial z}\]

    \begin{tcolorbox}[breakable, size=fbox, boxrule=1pt, pad at break*=1mm,colback=cellbackground, colframe=cellborder]
\prompt{In}{incolor}{5}{\boxspacing}
\begin{Verbatim}[commandchars=\\\{\}]
\PY{n}{z} \PY{o}{=} \PY{n}{sp}\PY{o}{.}\PY{n}{Symbol}\PY{p}{(}\PY{l+s+s2}{\PYZdq{}}\PY{l+s+s2}{z}\PY{l+s+s2}{\PYZdq{}}\PY{p}{)}
\PY{n}{phi} \PY{o}{=} \PY{n}{sp}\PY{o}{.}\PY{n}{Symbol}\PY{p}{(}\PY{l+s+s2}{\PYZdq{}}\PY{l+s+s2}{phi}\PY{l+s+s2}{\PYZdq{}}\PY{p}{)}
\PY{n}{p} \PY{o}{=} \PY{n}{sp}\PY{o}{.}\PY{n}{Symbol}\PY{p}{(}\PY{l+s+s2}{\PYZdq{}}\PY{l+s+s2}{rho}\PY{l+s+s2}{\PYZdq{}}\PY{p}{)}

\PY{n}{Dp} \PY{o}{=} \PY{p}{(}\PY{o}{\PYZhy{}}\PY{l+m+mi}{20}\PY{o}{/}\PY{n}{p}\PY{o}{*}\PY{o}{*}\PY{l+m+mi}{2}\PY{p}{)}\PY{o}{*}\PY{p}{(}\PY{n}{sp}\PY{o}{.}\PY{n}{sin}\PY{p}{(}\PY{n}{phi}\PY{p}{)}\PY{p}{)}\PY{o}{*}\PY{o}{*}\PY{l+m+mi}{2}
\PY{n}{Dphi} \PY{o}{=} \PY{p}{(}\PY{l+m+mi}{20}\PY{o}{/}\PY{n}{p}\PY{o}{*}\PY{o}{*}\PY{l+m+mi}{2}\PY{p}{)}\PY{o}{*}\PY{p}{(}\PY{n}{sp}\PY{o}{.}\PY{n}{sin}\PY{p}{(}\PY{l+m+mi}{2}\PY{o}{*}\PY{n}{phi}\PY{p}{)}\PY{p}{)}

\PY{n}{D} \PY{o}{=} \PY{p}{(}\PY{l+m+mi}{1}\PY{o}{/}\PY{n}{p}\PY{p}{)}\PY{o}{*}\PY{n}{sp}\PY{o}{.}\PY{n}{diff}\PY{p}{(}\PY{n}{p}\PY{o}{*}\PY{n}{Dp}\PY{p}{,}\PY{n}{p}\PY{p}{)} \PY{o}{+} \PY{p}{(}\PY{l+m+mi}{1}\PY{o}{/}\PY{n}{p}\PY{p}{)}\PY{o}{*}\PY{n}{sp}\PY{o}{.}\PY{n}{diff}\PY{p}{(}\PY{n}{Dphi}\PY{p}{,}\PY{n}{phi}\PY{p}{)}
\PY{n}{D}\PY{o}{.}\PY{n}{simplify}\PY{p}{(}\PY{p}{)}
\end{Verbatim}
\end{tcolorbox}
 
            
\prompt{Out}{outcolor}{5}{}
    
    $$\displaystyle \frac{20 \left(2 - 3 \sin^{2}{\left(\phi \right)}\right)}{\rho^{3}}$$

    

    tendo o divergente, pode-se qualquer a carga

\[ Q_{int} = \int^1_0\int^{\pi/2}_0\int^2_1 \frac{20(2-3\sin^2(\phi))}{\rho^3} \rho d\rho d\phi dz\]

    \begin{tcolorbox}[breakable, size=fbox, boxrule=1pt, pad at break*=1mm,colback=cellbackground, colframe=cellborder]
\prompt{In}{incolor}{6}{\boxspacing}
\begin{Verbatim}[commandchars=\\\{\}]
\PY{n}{f} \PY{o}{=} \PY{l+m+mi}{20}\PY{o}{*}\PY{p}{(}\PY{l+m+mi}{2}\PY{o}{\PYZhy{}}\PY{l+m+mi}{3}\PY{o}{*}\PY{p}{(}\PY{n}{sp}\PY{o}{.}\PY{n}{sin}\PY{p}{(}\PY{n}{phi}\PY{p}{)}\PY{p}{)}\PY{o}{*}\PY{o}{*}\PY{l+m+mi}{2}\PY{p}{)}\PY{o}{/}\PY{n}{p}\PY{o}{*}\PY{o}{*}\PY{l+m+mi}{2}

\PY{n}{Q} \PY{o}{=} \PY{n}{sp}\PY{o}{.}\PY{n}{integrate}\PY{p}{(}\PY{n}{sp}\PY{o}{.}\PY{n}{integrate}\PY{p}{(}\PY{n}{sp}\PY{o}{.}\PY{n}{integrate}\PY{p}{(}\PY{n}{f}\PY{p}{,}\PY{p}{(}\PY{n}{p}\PY{p}{,}\PY{l+m+mi}{1}\PY{p}{,}\PY{l+m+mi}{2}\PY{p}{)}\PY{p}{)}\PY{p}{,}\PY{p}{(}\PY{n}{phi}\PY{p}{,}\PY{l+m+mi}{0}\PY{p}{,}\PY{n}{sp}\PY{o}{.}\PY{n}{pi}\PY{o}{/}\PY{l+m+mi}{2}\PY{p}{)}\PY{p}{)}\PY{p}{,}\PY{p}{(}\PY{n}{z}\PY{p}{,}\PY{l+m+mi}{0}\PY{p}{,}\PY{l+m+mi}{1}\PY{p}{)}\PY{p}{)}
\PY{n}{Q}
\end{Verbatim}
\end{tcolorbox}
 
            
\prompt{Out}{outcolor}{6}{}
    
    $$\displaystyle \frac{5 \pi}{2}$$

    

    \[ Q_{int} = \frac{5\pi}{2} \ C\]

    \hypertarget{q7}{%
\subsection{Q7}\label{q7}}

    \begin{tcolorbox}[breakable, size=fbox, boxrule=1pt, pad at break*=1mm,colback=cellbackground, colframe=cellborder]
\prompt{In}{incolor}{7}{\boxspacing}
\begin{Verbatim}[commandchars=\\\{\}]
\PY{k+kn}{from} \PY{n+nn}{IPython}\PY{n+nn}{.}\PY{n+nn}{display} \PY{k+kn}{import} \PY{n}{Image}
\PY{n}{Image}\PY{p}{(}\PY{l+s+s2}{\PYZdq{}}\PY{l+s+s2}{esfera.png}\PY{l+s+s2}{\PYZdq{}}\PY{p}{)}
\end{Verbatim}
\end{tcolorbox}
 
            
\prompt{Out}{outcolor}{7}{}
    
    \begin{center}
    \adjustimage{max size={0.9\linewidth}{0.9\paperheight}}{output_38_0.png}
    \end{center}
    { \hspace*{\fill} \\}
    

    Planificando para achar os limites de integração, aplicando geometria
básica consegue-se achar o limite de \(\sqrt{3}\)

    \begin{tcolorbox}[breakable, size=fbox, boxrule=1pt, pad at break*=1mm,colback=cellbackground, colframe=cellborder]
\prompt{In}{incolor}{8}{\boxspacing}
\begin{Verbatim}[commandchars=\\\{\}]
\PY{k+kn}{from} \PY{n+nn}{IPython}\PY{n+nn}{.}\PY{n+nn}{display} \PY{k+kn}{import} \PY{n}{Image}
\PY{n}{Image}\PY{p}{(}\PY{l+s+s2}{\PYZdq{}}\PY{l+s+s2}{plano.png}\PY{l+s+s2}{\PYZdq{}}\PY{p}{)}
\end{Verbatim}
\end{tcolorbox}
 
            
\prompt{Out}{outcolor}{8}{}
    
    \begin{center}
    \adjustimage{max size={0.9\linewidth}{0.9\paperheight}}{output_40_0.png}
    \end{center}
    { \hspace*{\fill} \\}
    

    Sabe-se que $\psi = Q\_\{int\} = Q\_\{linha\} + Q\_\{plano\} $ e

$$\begin{cases} 
\displaystyle Q_{linha} = \int^2_{-2} \rho_L dl = 16\pi \ C \\ \\
\displaystyle Q_{plano} = \iint \rho_s dS = \int^{2\pi}_0 \int^{\sqrt{3}}_{0} 20\rho d\rho d\phi = 60\pi \ C 
\end{cases}$$

logo o fluxo é \[\psi = Q_{int} = 76\pi \ C \]

    \hypertarget{q8}{%
\subsection{Q8}\label{q8}}

    \[ \psi = \oint \vec{D} \ d\vec{s} \]

$$\begin{cases}
\displaystyle \vec{D} = \frac{Q}{4\pi r^2} \vec{a}_r \\ \\
\displaystyle d\vec{s} = r^2\sin{\theta} \ d\phi \ d\theta \ \vec{a}_r 
\end{cases}$$

\[\displaystyle \psi = \int^{\beta}_{\alpha} \int^{\pi}_0 \frac{Q}{4\pi r^2} r^2\sin{\theta} \ d\theta \ d\phi \]

    \begin{tcolorbox}[breakable, size=fbox, boxrule=1pt, pad at break*=1mm,colback=cellbackground, colframe=cellborder]
\prompt{In}{incolor}{9}{\boxspacing}
\begin{Verbatim}[commandchars=\\\{\}]
\PY{n}{B} \PY{o}{=} \PY{n}{sp}\PY{o}{.}\PY{n}{Symbol}\PY{p}{(}\PY{l+s+s2}{\PYZdq{}}\PY{l+s+s2}{beta}\PY{l+s+s2}{\PYZdq{}}\PY{p}{)}
\PY{n}{a} \PY{o}{=} \PY{n}{sp}\PY{o}{.}\PY{n}{Symbol}\PY{p}{(}\PY{l+s+s2}{\PYZdq{}}\PY{l+s+s2}{alpha}\PY{l+s+s2}{\PYZdq{}}\PY{p}{)}
\PY{n}{r} \PY{o}{=} \PY{n}{sp}\PY{o}{.}\PY{n}{Symbol}\PY{p}{(}\PY{l+s+s2}{\PYZdq{}}\PY{l+s+s2}{r}\PY{l+s+s2}{\PYZdq{}}\PY{p}{)}
\PY{n}{o} \PY{o}{=} \PY{n}{sp}\PY{o}{.}\PY{n}{Symbol}\PY{p}{(}\PY{l+s+s2}{\PYZdq{}}\PY{l+s+s2}{theta}\PY{l+s+s2}{\PYZdq{}}\PY{p}{)}
\PY{n}{Q} \PY{o}{=} \PY{n}{sp}\PY{o}{.}\PY{n}{Symbol}\PY{p}{(}\PY{l+s+s2}{\PYZdq{}}\PY{l+s+s2}{Q}\PY{l+s+s2}{\PYZdq{}}\PY{p}{)}
\PY{n}{phi} \PY{o}{=} \PY{n}{sp}\PY{o}{.}\PY{n}{Symbol}\PY{p}{(}\PY{l+s+s2}{\PYZdq{}}\PY{l+s+s2}{phi}\PY{l+s+s2}{\PYZdq{}}\PY{p}{)}

\PY{n}{f} \PY{o}{=} \PY{p}{(}\PY{n}{sp}\PY{o}{.}\PY{n}{sin}\PY{p}{(}\PY{n}{o}\PY{p}{)}\PY{o}{*}\PY{n}{Q}\PY{p}{)}\PY{o}{/}\PY{p}{(}\PY{l+m+mi}{4}\PY{o}{*}\PY{n}{sp}\PY{o}{.}\PY{n}{pi}\PY{p}{)}

\PY{n}{psi} \PY{o}{=} \PY{n}{sp}\PY{o}{.}\PY{n}{integrate}\PY{p}{(}\PY{n}{sp}\PY{o}{.}\PY{n}{integrate}\PY{p}{(}\PY{n}{f}\PY{p}{,}\PY{p}{(}\PY{n}{o}\PY{p}{,}\PY{l+m+mi}{0}\PY{p}{,}\PY{n}{sp}\PY{o}{.}\PY{n}{pi}\PY{p}{)}\PY{p}{)}\PY{p}{,}\PY{p}{(}\PY{n}{phi}\PY{p}{,}\PY{n}{a}\PY{p}{,}\PY{n}{B}\PY{p}{)}\PY{p}{)}
\PY{n}{psi}\PY{o}{.}\PY{n}{simplify}\PY{p}{(}\PY{p}{)}
\end{Verbatim}
\end{tcolorbox}
 
            
\prompt{Out}{outcolor}{9}{}
    
    $$\displaystyle \frac{Q \left(- \alpha + \beta\right)}{2 \pi}$$

    

    \hypertarget{q9}{%
\subsection{Q9}\label{q9}}

    \hypertarget{a}{%
\subsubsection{a)}\label{a}}

    \begin{tcolorbox}[breakable, size=fbox, boxrule=1pt, pad at break*=1mm,colback=cellbackground, colframe=cellborder]
\prompt{In}{incolor}{10}{\boxspacing}
\begin{Verbatim}[commandchars=\\\{\}]
\PY{k+kn}{from} \PY{n+nn}{IPython}\PY{n+nn}{.}\PY{n+nn}{display} \PY{k+kn}{import} \PY{n}{Image}
\PY{n}{Image}\PY{p}{(}\PY{l+s+s2}{\PYZdq{}}\PY{l+s+s2}{A.png}\PY{l+s+s2}{\PYZdq{}}\PY{p}{)}
\end{Verbatim}
\end{tcolorbox}
 
            
\prompt{Out}{outcolor}{10}{}
    
    \begin{center}
    \adjustimage{max size={0.9\linewidth}{0.9\paperheight}}{output_47_0.png}
    \end{center}
    { \hspace*{\fill} \\}
    

    Considerou-se o fluxo de desindade pelos três elementos de distribuição,
tal que :
\[ \vec{D} = \vec{D}_{carga}+\vec{D}_{linha}+\vec{D}_{plano} \]

$$\begin{cases}
\\
\displaystyle \vec{D}_{carga} = \frac{Q}{4\pi r^2 } \vec{a}_z\\  \\
\displaystyle \vec{D}_{linha} = \frac{\rho_L}{2\pi\rho} \vec{a}_z \\  \\
\displaystyle \vec{D}_{plano} = \frac{\rho_s}{2} \vec{a}_z 
\end{cases}$$

    \begin{tcolorbox}[breakable, size=fbox, boxrule=1pt, pad at break*=1mm,colback=cellbackground, colframe=cellborder]
\prompt{In}{incolor}{11}{\boxspacing}
\begin{Verbatim}[commandchars=\\\{\}]
\PY{n}{Q}\PY{o}{=}\PY{l+m+mf}{6e\PYZhy{}9}
\PY{n}{rhoL} \PY{o}{=} \PY{l+m+mf}{180e\PYZhy{}9}
\PY{n}{rhoS} \PY{o}{=} \PY{l+m+mf}{25e\PYZhy{}9}
\PY{n}{r} \PY{o}{=} \PY{l+m+mi}{4}
\PY{n}{rho} \PY{o}{=} \PY{l+m+mi}{4}

\PY{n}{Dq} \PY{o}{=} \PY{n}{Q}\PY{o}{/}\PY{p}{(}\PY{l+m+mi}{4}\PY{o}{*}\PY{n}{sp}\PY{o}{.}\PY{n}{pi}\PY{o}{*}\PY{n}{r}\PY{o}{*}\PY{o}{*}\PY{l+m+mi}{2}\PY{p}{)}

\PY{n}{Dl} \PY{o}{=} \PY{n}{rhoL}\PY{o}{/}\PY{p}{(}\PY{l+m+mi}{2}\PY{o}{*}\PY{n}{sp}\PY{o}{.}\PY{n}{pi}\PY{o}{*}\PY{n}{rho}\PY{p}{)}

\PY{n}{Dp} \PY{o}{=} \PY{n}{rhoS}\PY{o}{/}\PY{l+m+mi}{2}


\PY{n}{D} \PY{o}{=} \PY{n}{Dq}\PY{o}{+}\PY{n}{Dl}\PY{o}{+}\PY{n}{Dp}

\PY{n}{D}\PY{o}{.}\PY{n}{evalf}\PY{p}{(}\PY{l+m+mi}{4}\PY{p}{)}
\end{Verbatim}
\end{tcolorbox}
 
            
\prompt{Out}{outcolor}{11}{}
    
    $$\displaystyle 1.969 \cdot 10^{-8}$$

    

    \[ \vec{D} = 0.1969 \ \vec{a}_z \]

    \hypertarget{b}{%
\subsubsection{b)}\label{b}}

precisa-se arrumar a parte vetorial, temos


$$\begin{cases}
\text{Carga } \rightarrow  \displaystyle \vec{a}_r =  \frac{\vec{a}_x+2\vec{a}_y+4\vec{a}_z}{\sqrt{21}} \\ \\
\text{Linha }\rightarrow  \displaystyle \vec{a}_\rho = \frac{2\vec{a}_y+4\vec{a}_z}{\sqrt{20}} \\ \\
\text{Plano } \rightarrow \displaystyle \vec{a}_n = \vec{a}_z 
\end{cases}$$

    \begin{tcolorbox}[breakable, size=fbox, boxrule=1pt, pad at break*=1mm,colback=cellbackground, colframe=cellborder]
\prompt{In}{incolor}{12}{\boxspacing}
\begin{Verbatim}[commandchars=\\\{\}]
\PY{n}{r} \PY{o}{=} \PY{n}{np}\PY{o}{.}\PY{n}{array}\PY{p}{(}\PY{p}{[}\PY{l+m+mi}{1}\PY{p}{,}\PY{l+m+mi}{2}\PY{p}{,}\PY{l+m+mi}{4}\PY{p}{]}\PY{p}{)}
\PY{n}{R} \PY{o}{=} \PY{n}{np}\PY{o}{.}\PY{n}{sqrt}\PY{p}{(}\PY{n+nb}{sum}\PY{p}{(}\PY{n}{r}\PY{o}{*}\PY{o}{*}\PY{l+m+mi}{2}\PY{p}{)}\PY{p}{)}
\PY{n}{ar} \PY{o}{=} \PY{n}{r}\PY{o}{/}\PY{n}{R}

\PY{n}{p} \PY{o}{=} \PY{n}{np}\PY{o}{.}\PY{n}{array}\PY{p}{(}\PY{p}{[}\PY{l+m+mi}{0}\PY{p}{,}\PY{l+m+mi}{2}\PY{p}{,}\PY{l+m+mi}{4}\PY{p}{]}\PY{p}{)}
\PY{n}{P} \PY{o}{=} \PY{n}{np}\PY{o}{.}\PY{n}{sqrt}\PY{p}{(}\PY{n+nb}{sum}\PY{p}{(}\PY{n}{p}\PY{o}{*}\PY{o}{*}\PY{l+m+mi}{2}\PY{p}{)}\PY{p}{)}
\PY{n}{ap} \PY{o}{=} \PY{n}{p}\PY{o}{/}\PY{n}{P}

\PY{n}{Dq} \PY{o}{=} \PY{p}{(}\PY{n}{Q}\PY{o}{/}\PY{p}{(}\PY{l+m+mi}{4}\PY{o}{*}\PY{n}{pi}\PY{o}{*}\PY{n}{R}\PY{o}{*}\PY{o}{*}\PY{l+m+mi}{2}\PY{p}{)}\PY{p}{)}\PY{o}{*}\PY{n}{ar}

\PY{n}{Dl} \PY{o}{=} \PY{p}{(}\PY{n}{rhoL}\PY{o}{/}\PY{p}{(}\PY{l+m+mi}{2}\PY{o}{*}\PY{n}{pi}\PY{o}{*}\PY{n}{P}\PY{p}{)}\PY{p}{)}\PY{o}{*}\PY{n}{ap}

\PY{n}{Dp} \PY{o}{=} \PY{p}{(}\PY{n}{rhoS}\PY{o}{/}\PY{l+m+mi}{2}\PY{p}{)}\PY{o}{*}\PY{n}{np}\PY{o}{.}\PY{n}{array}\PY{p}{(}\PY{p}{[}\PY{l+m+mi}{0}\PY{p}{,}\PY{l+m+mi}{0}\PY{p}{,}\PY{l+m+mi}{1}\PY{p}{]}\PY{p}{)}

\PY{n}{D} \PY{o}{=} \PY{n}{Dq}\PY{o}{+}\PY{n}{Dl}\PY{o}{+}\PY{n}{Dp}

\PY{n+nb}{print}\PY{p}{(}\PY{l+s+s2}{\PYZdq{}}\PY{l+s+s2}{Dq =  }\PY{l+s+si}{\PYZob{}:.3e\PYZcb{}}\PY{l+s+s2}{ax }\PY{l+s+se}{\PYZbs{}t}\PY{l+s+s2}{ }\PY{l+s+si}{\PYZob{}:.3e\PYZcb{}}\PY{l+s+s2}{ay }\PY{l+s+se}{\PYZbs{}t}\PY{l+s+s2}{ }\PY{l+s+si}{\PYZob{}:.3e\PYZcb{}}\PY{l+s+s2}{az }\PY{l+s+s2}{\PYZdq{}}\PY{o}{.}\PY{n}{format}\PY{p}{(}\PY{o}{*}\PY{n}{Dq}\PY{p}{)}\PY{p}{)}
\PY{n+nb}{print}\PY{p}{(}\PY{l+s+s2}{\PYZdq{}}\PY{l+s+s2}{Dl = }\PY{l+s+si}{\PYZob{}:.3e\PYZcb{}}\PY{l+s+s2}{ax }\PY{l+s+se}{\PYZbs{}t}\PY{l+s+s2}{ }\PY{l+s+si}{\PYZob{}:.3e\PYZcb{}}\PY{l+s+s2}{ay }\PY{l+s+se}{\PYZbs{}t}\PY{l+s+s2}{ }\PY{l+s+si}{\PYZob{}:.3e\PYZcb{}}\PY{l+s+s2}{az }\PY{l+s+s2}{\PYZdq{}}\PY{o}{.}\PY{n}{format}\PY{p}{(}\PY{o}{*}\PY{n}{Dl}\PY{p}{)}\PY{p}{)}
\PY{n+nb}{print}\PY{p}{(}\PY{l+s+s2}{\PYZdq{}}\PY{l+s+s2}{Dp = }\PY{l+s+si}{\PYZob{}:.3e\PYZcb{}}\PY{l+s+s2}{ax }\PY{l+s+se}{\PYZbs{}t}\PY{l+s+s2}{ }\PY{l+s+si}{\PYZob{}:.3e\PYZcb{}}\PY{l+s+s2}{ay }\PY{l+s+se}{\PYZbs{}t}\PY{l+s+s2}{ }\PY{l+s+si}{\PYZob{}:.3e\PYZcb{}}\PY{l+s+s2}{az }\PY{l+s+se}{\PYZbs{}n}\PY{l+s+se}{\PYZbs{}n}\PY{l+s+s2}{\PYZdq{}}\PY{o}{.}\PY{n}{format}\PY{p}{(}\PY{o}{*}\PY{n}{Dp}\PY{p}{)}\PY{p}{)}

\PY{n+nb}{print}\PY{p}{(}\PY{l+s+s2}{\PYZdq{}}\PY{l+s+s2}{D = }\PY{l+s+si}{\PYZob{}:.3e\PYZcb{}}\PY{l+s+s2}{ax }\PY{l+s+se}{\PYZbs{}t}\PY{l+s+s2}{ }\PY{l+s+si}{\PYZob{}:.3e\PYZcb{}}\PY{l+s+s2}{ay }\PY{l+s+se}{\PYZbs{}t}\PY{l+s+s2}{ }\PY{l+s+si}{\PYZob{}:.3e\PYZcb{}}\PY{l+s+s2}{az }\PY{l+s+s2}{\PYZdq{}}\PY{o}{.}\PY{n}{format}\PY{p}{(}\PY{o}{*}\PY{n}{D}\PY{p}{)}\PY{p}{)}
\end{Verbatim}
\end{tcolorbox}

    \begin{Verbatim}[commandchars=\\\{\}]
Dq =  4.961e-12ax        9.923e-12ay     1.985e-11az
Dl = 0.000e+00ax         2.865e-09ay     5.730e-09az
Dp = 0.000e+00ax         0.000e+00ay     1.250e-08az


D = 4.961e-12ax          2.875e-09ay     1.825e-08az
    \end{Verbatim}

    \hypertarget{c}{%
\subsubsection{c)}\label{c}}

$$    \begin{cases}
\text{Linha } \rightarrow \displaystyle Q_L = \int^4_{-4} \rho_L dx =  1440 \ nC\\ \\
\text{Plano } \rightarrow \displaystyle Q_p = \int^{2\pi}_0 \int^4_{0} \rho_s \ \rho d\rho d\phi =  400\pi \ nC
\end{cases}$$

    

    \[ Q = Q_L + Q_p = 8.7\ \mu C\]

    \hypertarget{q10}{%
\subsection{Q10}\label{q10}}

    Calculo do divergente
\[ \vec{\nabla}\cdot\vec{D} = \frac{1}{\rho} \frac{\partial (\rho D_{\rho})}{\partial \rho}+ \frac{1}{\rho} \frac{\partial (D_{\phi})}{\partial \phi} + \frac{\partial (D_{z})}{\partial z}\]

    \begin{tcolorbox}[breakable, size=fbox, boxrule=1pt, pad at break*=1mm,colback=cellbackground, colframe=cellborder]
\prompt{In}{incolor}{13}{\boxspacing}
\begin{Verbatim}[commandchars=\\\{\}]
\PY{n}{z} \PY{o}{=} \PY{n}{sp}\PY{o}{.}\PY{n}{Symbol}\PY{p}{(}\PY{l+s+s2}{\PYZdq{}}\PY{l+s+s2}{z}\PY{l+s+s2}{\PYZdq{}}\PY{p}{)}
\PY{n}{p} \PY{o}{=} \PY{n}{sp}\PY{o}{.}\PY{n}{Symbol}\PY{p}{(}\PY{l+s+s2}{\PYZdq{}}\PY{l+s+s2}{rho}\PY{l+s+s2}{\PYZdq{}}\PY{p}{)}

\PY{n}{Dp} \PY{o}{=} \PY{l+m+mi}{30}\PY{o}{*}\PY{n}{sp}\PY{o}{.}\PY{n}{exp}\PY{p}{(}\PY{o}{\PYZhy{}}\PY{n}{p}\PY{p}{)}
\PY{n}{Dz} \PY{o}{=} \PY{o}{\PYZhy{}}\PY{l+m+mi}{2}\PY{o}{*}\PY{n}{z}

\PY{n}{D} \PY{o}{=} \PY{p}{(}\PY{l+m+mi}{1}\PY{o}{/}\PY{n}{p}\PY{p}{)}\PY{o}{*}\PY{p}{(}\PY{n}{sp}\PY{o}{.}\PY{n}{diff}\PY{p}{(}\PY{n}{p}\PY{o}{*}\PY{n}{Dp}\PY{p}{,}\PY{n}{p}\PY{p}{)}\PY{p}{)} \PY{o}{+} \PY{n}{sp}\PY{o}{.}\PY{n}{diff}\PY{p}{(}\PY{n}{Dz}\PY{p}{,}\PY{n}{z}\PY{p}{)}
\PY{n}{D}
\end{Verbatim}
\end{tcolorbox}
 
            
\prompt{Out}{outcolor}{13}{}
    
    $$\displaystyle -2 + \frac{- 30 \rho e^{- \rho} + 30 e^{- \rho}}{\rho}$$

    

    integrando:
\[ \int \vec{\nabla}\cdot\vec{D} dv = \int^5_0 \int^{2\pi}_{0} \int^2_0 \vec{\nabla}\cdot\vec{D} \ \rho d\rho d\phi dz \]

    \begin{tcolorbox}[breakable, size=fbox, boxrule=1pt, pad at break*=1mm,colback=cellbackground, colframe=cellborder]
\prompt{In}{incolor}{ }{\boxspacing}
\begin{Verbatim}[commandchars=\\\{\}]

\end{Verbatim}
\end{tcolorbox}

    \begin{tcolorbox}[breakable, size=fbox, boxrule=1pt, pad at break*=1mm,colback=cellbackground, colframe=cellborder]
\prompt{In}{incolor}{14}{\boxspacing}
\begin{Verbatim}[commandchars=\\\{\}]
\PY{n}{d}\PY{o}{=}\PY{n}{sp}\PY{o}{.}\PY{n}{integrate}\PY{p}{(}\PY{n}{sp}\PY{o}{.}\PY{n}{integrate}\PY{p}{(}\PY{n}{sp}\PY{o}{.}\PY{n}{integrate}\PY{p}{(}\PY{n}{D}\PY{o}{*}\PY{n}{p}\PY{p}{,}\PY{p}{(}\PY{n}{p}\PY{p}{,}\PY{l+m+mi}{0}\PY{p}{,}\PY{l+m+mi}{2}\PY{p}{)}\PY{p}{)}\PY{p}{,}\PY{p}{(}\PY{n}{phi}\PY{p}{,}\PY{l+m+mi}{0}\PY{p}{,}\PY{l+m+mi}{2}\PY{o}{*}\PY{n}{sp}\PY{o}{.}\PY{n}{pi}\PY{p}{)}\PY{p}{)}\PY{p}{,}\PY{p}{(}\PY{n}{z}\PY{p}{,}\PY{l+m+mi}{0}\PY{p}{,}\PY{l+m+mi}{5}\PY{p}{)}\PY{p}{)}
\PY{n}{d}
\end{Verbatim}
\end{tcolorbox}
 
            
\prompt{Out}{outcolor}{14}{}
    
    $$\displaystyle 10 \pi \left(-4 + \frac{60}{e^{2}}\right)$$

    

    \begin{tcolorbox}[breakable, size=fbox, boxrule=1pt, pad at break*=1mm,colback=cellbackground, colframe=cellborder]
\prompt{In}{incolor}{15}{\boxspacing}
\begin{Verbatim}[commandchars=\\\{\}]
\PY{n}{d}\PY{o}{.}\PY{n}{evalf}\PY{p}{(}\PY{l+m+mi}{6}\PY{p}{)}
\end{Verbatim}
\end{tcolorbox}
 
            
\prompt{Out}{outcolor}{15}{}
    
    $$\displaystyle 129.437$$

    

    \[ \int \vec{\nabla}\cdot\vec{D} dv = 129.437 \ C \]

    \hypertarget{q11}{%
\subsection{Q11}\label{q11}}

    \hypertarget{a}{%
\subsubsection{a)}\label{a}}

    \begin{tcolorbox}[breakable, size=fbox, boxrule=1pt, pad at break*=1mm,colback=cellbackground, colframe=cellborder]
\prompt{In}{incolor}{16}{\boxspacing}
\begin{Verbatim}[commandchars=\\\{\}]
\PY{n}{x} \PY{o}{=} \PY{n}{sp}\PY{o}{.}\PY{n}{Symbol}\PY{p}{(}\PY{l+s+s2}{\PYZdq{}}\PY{l+s+s2}{x}\PY{l+s+s2}{\PYZdq{}}\PY{p}{)}
\PY{n}{z} \PY{o}{=} \PY{n}{sp}\PY{o}{.}\PY{n}{Symbol}\PY{p}{(}\PY{l+s+s2}{\PYZdq{}}\PY{l+s+s2}{z}\PY{l+s+s2}{\PYZdq{}}\PY{p}{)}
\PY{n}{y} \PY{o}{=} \PY{n}{sp}\PY{o}{.}\PY{n}{Symbol}\PY{p}{(}\PY{l+s+s2}{\PYZdq{}}\PY{l+s+s2}{y}\PY{l+s+s2}{\PYZdq{}}\PY{p}{)}


\PY{n}{D} \PY{o}{=} \PY{l+m+mi}{3}\PY{o}{*}\PY{n}{x}\PY{o}{*}\PY{o}{*}\PY{l+m+mi}{2} \PY{o}{+} \PY{l+m+mi}{2}\PY{o}{*}\PY{n}{z} \PY{o}{+} \PY{n}{x}\PY{o}{*}\PY{o}{*}\PY{l+m+mi}{2} \PY{o}{*}\PY{n}{z}

\PY{n}{Div} \PY{o}{=} \PY{n}{sp}\PY{o}{.}\PY{n}{Derivative}\PY{p}{(}\PY{n}{D}\PY{p}{,}\PY{n}{x}\PY{p}{)} \PY{o}{+} \PY{n}{sp}\PY{o}{.}\PY{n}{Derivative}\PY{p}{(}\PY{n}{D}\PY{p}{,}\PY{n}{z}\PY{p}{)}
\PY{n}{Div}
\end{Verbatim}
\end{tcolorbox}
 
            
\prompt{Out}{outcolor}{16}{}
    
    $$\displaystyle \frac{\partial}{\partial x} \left(x^{2} z + 3 x^{2} + 2 z\right) + \frac{\partial}{\partial z} \left(x^{2} z + 3 x^{2} + 2 z\right)$$

    

    \begin{tcolorbox}[breakable, size=fbox, boxrule=1pt, pad at break*=1mm,colback=cellbackground, colframe=cellborder]
\prompt{In}{incolor}{17}{\boxspacing}
\begin{Verbatim}[commandchars=\\\{\}]
\PY{n}{Div}\PY{o}{.}\PY{n}{doit}\PY{p}{(}\PY{p}{)}
\end{Verbatim}
\end{tcolorbox}
 
            
\prompt{Out}{outcolor}{17}{}
    
    $$\displaystyle x^{2} + 2 x z + 6 x + 2$$

    

    \begin{tcolorbox}[breakable, size=fbox, boxrule=1pt, pad at break*=1mm,colback=cellbackground, colframe=cellborder]
\prompt{In}{incolor}{18}{\boxspacing}
\begin{Verbatim}[commandchars=\\\{\}]
\PY{n}{Int} \PY{o}{=} \PY{n}{sp}\PY{o}{.}\PY{n}{Integral}\PY{p}{(}\PY{n}{sp}\PY{o}{.}\PY{n}{Integral}\PY{p}{(}\PY{n}{sp}\PY{o}{.}\PY{n}{Integral}\PY{p}{(}\PY{n}{Div}\PY{p}{,}\PY{p}{(}\PY{n}{x}\PY{p}{,}\PY{o}{\PYZhy{}}\PY{l+m+mi}{2}\PY{p}{,}\PY{l+m+mi}{2}\PY{p}{)}\PY{p}{)}\PY{p}{,}\PY{p}{(}\PY{n}{y}\PY{p}{,}\PY{o}{\PYZhy{}}\PY{l+m+mi}{2}\PY{p}{,}\PY{l+m+mi}{2}\PY{p}{)}\PY{p}{)}\PY{p}{,}\PY{p}{(}\PY{n}{z}\PY{p}{,}\PY{o}{\PYZhy{}}\PY{l+m+mi}{2}\PY{p}{,}\PY{l+m+mi}{2}\PY{p}{)}\PY{p}{)}
\PY{n}{Int}
\end{Verbatim}
\end{tcolorbox}
 
            
\prompt{Out}{outcolor}{18}{}
    
    $$\displaystyle \int\limits_{-2}^{2}\int\limits_{-2}^{2}\int\limits_{-2}^{2} \left(\frac{\partial}{\partial x} \left(x^{2} z + 3 x^{2} + 2 z\right) + \frac{\partial}{\partial z} \left(x^{2} z + 3 x^{2} + 2 z\right)\right)\, dx\, dy\, dz$$

    

    \begin{tcolorbox}[breakable, size=fbox, boxrule=1pt, pad at break*=1mm,colback=cellbackground, colframe=cellborder]
\prompt{In}{incolor}{19}{\boxspacing}
\begin{Verbatim}[commandchars=\\\{\}]
\PY{n}{Int}\PY{o}{.}\PY{n}{doit}\PY{p}{(}\PY{p}{)}\PY{o}{.}\PY{n}{evalf}\PY{p}{(}\PY{l+m+mi}{6}\PY{p}{)}
\end{Verbatim}
\end{tcolorbox}
 
            
\prompt{Out}{outcolor}{19}{}
    
    $$\displaystyle 213.333$$

    

    \hypertarget{b}{%
\subsubsection{b)}\label{b}}

    \begin{tcolorbox}[breakable, size=fbox, boxrule=1pt, pad at break*=1mm,colback=cellbackground, colframe=cellborder]
\prompt{In}{incolor}{20}{\boxspacing}
\begin{Verbatim}[commandchars=\\\{\}]
\PY{n}{IntB} \PY{o}{=} \PY{n}{sp}\PY{o}{.}\PY{n}{Integral}\PY{p}{(}\PY{n}{sp}\PY{o}{.}\PY{n}{Integral}\PY{p}{(}\PY{n}{sp}\PY{o}{.}\PY{n}{Integral}\PY{p}{(}\PY{n}{Div}\PY{p}{,}\PY{p}{(}\PY{n}{x}\PY{p}{,}\PY{l+m+mi}{0}\PY{p}{,}\PY{l+m+mi}{4}\PY{p}{)}\PY{p}{)}\PY{p}{,}\PY{p}{(}\PY{n}{y}\PY{p}{,}\PY{l+m+mi}{0}\PY{p}{,}\PY{l+m+mi}{4}\PY{p}{)}\PY{p}{)}\PY{p}{,}\PY{p}{(}\PY{n}{z}\PY{p}{,}\PY{l+m+mi}{0}\PY{p}{,}\PY{l+m+mi}{4}\PY{p}{)}\PY{p}{)}
\PY{n}{IntB}
\end{Verbatim}
\end{tcolorbox}
 
            
\prompt{Out}{outcolor}{20}{}
    
    $$\displaystyle \int\limits_{0}^{4}\int\limits_{0}^{4}\int\limits_{0}^{4} \left(\frac{\partial}{\partial x} \left(x^{2} z + 3 x^{2} + 2 z\right) + \frac{\partial}{\partial z} \left(x^{2} z + 3 x^{2} + 2 z\right)\right)\, dx\, dy\, dz$$

    

    \begin{tcolorbox}[breakable, size=fbox, boxrule=1pt, pad at break*=1mm,colback=cellbackground, colframe=cellborder]
\prompt{In}{incolor}{22}{\boxspacing}
\begin{Verbatim}[commandchars=\\\{\}]
\PY{n}{IntB}\PY{o}{.}\PY{n}{doit}\PY{p}{(}\PY{p}{)}\PY{o}{.}\PY{n}{evalf}\PY{p}{(}\PY{l+m+mi}{6}\PY{p}{)}
\end{Verbatim}
\end{tcolorbox}
 
            
\prompt{Out}{outcolor}{22}{}
    
    $$\displaystyle 1749.33$$

    


    % Add a bibliography block to the postdoc
    
    
    
\end{document}
